\documentclass[unicode,20pt]{beamer}
\usepackage[orientation=portrait,size=a0,final,scale=1.5]{beamerposter}
\usepackage{booktabs}
\usepackage{indent}
\usepackage{framed}
\newcommand\header[1]{\multicolumn{1}{c}{\textbf{#1}}}
\usetheme{sakuraposter}
\newfontfamily\exo{Exo2}
\ltjsetparameter{%
    jacharrange={%
        -2, % Greek and Cyrillic letters
        -3  % Punctuations and Miscellaneous symbols
    },
    alxspmode={`#,allow},
    alxspmode={`/,allow}
}
\usepackage[natbib,%
            backend=biber,
            hyperref=false,
            style=authoryear,
            maxcitenames=2,
            firstinits]{biblatex}
\ExecuteBibliographyOptions{doi=false,url=false}
\AtBeginBibliography{\footnotesize}
\addbibresource{reference.bib}
\newcommand\emoji[1]{\raisebox{-5pt}{\includegraphics[height=31pt]{emoji-#1}\hspace{.1em}}}
\newcommand\better[1]{\textcolor{sPink}{\textbf{#1}}}
\title{Evaluation Metrics for Automatically Generated Metaphorical Expressions}
\author{Akira Miyazawa and Yusuke Miyao}
\institute{The Graduate University for Advanced Studies / National Institute of Informatics}
\newcommand\msubsection[1]{\textcolor{sPink}{\textbf{\large \exo{#1}}}}
\newcommand\mettype[1]{\textcolor{sDarkBlue}{\textbf{#1}}}
\newcommand\myemph[1]{\textcolor{sPink}{\textbf{#1}}}
\hypersetup{%
    unicode,
    colorlinks=false,
    allcolors=sDarkBlue
}
\setbeamertemplate{headline}{%
    \leavevmode
    \begin{beamercolorbox}[wd=\paperwidth]{headline}
        \centering
        \vskip14mm
        \usebeamercolor{title in headline}{%
            \usebeamerfont{title in headline}\inserttitle
        }
        \par
        \vskip8mm
        \usebeamercolor{author in headline}{%
            \usebeamerfont{author in headline}\insertauthor
        }
        \par
        \vskip8mm
        \usebeamercolor{institute in headline}{%
            \usebeamerfont{institute in headline}\insertinstitute
        }
        \par
    \vskip14mm
    \end{beamercolorbox}
}

\begin{document}
\setlength\labelsep{\dimexpr\labelsep - 0.2em\relax}
\setlength\leftmargini{\dimexpr\leftmargini - 0.1em\relax}
\begin{frame}
    \vspace*{-2ex}
    \begin{columns}[t]
        \begin{column}{.47\textwidth}
            \begin{block}{INTRODUCTION}
                \textbf{Automatic generation of metaphors} helps
                us write novels, poems, etc.

                \bigskip

                Existing systems
                have focused on only \textbf{similes}
                such as ``\emph{T} like \emph{S}''
                \citep{enkitada2001,abe2006}.

                \bigskip

                We propose
                \myemph{metaphoricity},
                \myemph{novelty},
                \myemph{comprehensibility},
                and \myemph{overall evaluation}
                to get \textbf{``good metaphors''}.

                \bigskip


            \end{block}

            \begin{block}{EXPERIMENTAL SETTINGS}

                By \textbf{crowdsourcing},
                we got \textbf{10 scores}
                for each expression and metric.
                The scores were
                evaluated on a \textbf{five-point scale}.

                \bigskip
                \bigskip

                \begin{columns}[c]
                    \begin{column}{0.4\textwidth}
                        \raggedleft
                        \includegraphics[width=0.9\textwidth]{crowdsourcing2-crop.png}
                    \end{column}

                    \begin{column}{0.53\textwidth}
                        Do you feel that the expression is metaphorical?

                        \colorbox{sLightGray}{\textcolor{black}{%
                            \textbf{\emph{ai ga ahureru} (love overflows)}}}

                        \smallskip

                         \hspace*{.2em}5. It seems to be metaphorical.

                         \hspace*{.2em}1. It doesn't seem to be metaphorical.
                    \end{column}
                \end{columns}
            \end{block}

        \end{column}
        \begin{column}{.47\textwidth}

            \begin{block}{TARGETS}

                We made 1,360 Japanese expressions
                by combining 40 nouns and 34 verbal phrases
                following the method of \citet{ennabeshima2011}.

                \bigskip
                \bigskip

                \msubsection{Example}
                \begin{center}
                \emph{ai} (love) + \emph{X ga ahureru} (\emph{X} overflows)
                → \emph{ai ga ahureru} (love overflows)
                \end{center}

                \bigskip

                \msubsection{Nouns}

                \begin{itemize}
                    \item \mettype{FLUID AND SOLID}

                        \emph{mizu} (water) \emoji{u1f4a7},
                        \emph{suna} (sand) \emoji{u1f3dc},
                        \emph{iwa} (rock) \emoji{u26f0}, etc.

                    \item \mettype{EMOTION}

                        \emph{kimoti} (feeling) \emoji{u1f610},
                        \emph{ai} (love) \emoji{u1f60d},
                        \emph{tanosisa} (enjoyment) \emoji{u1f606},

                        \emph{kyouhu} (fear) \emoji{u1f628},
                        \emph{ikari} (anger) \emoji{u1f624},
                        \emph{kanasimi} (sorrow) \emoji{u1f622}, etc.
%                        \emph{zouo} (hatred) \emoji{u1f620},
%                        \emph{human} (complaint) \emoji{u1f612},

                    \item \mettype{IDEA}

                        \emph{rikai} (understanding) \emoji{u1f44d},
                        \emph{jouhou} (information) \emoji{u1f4be}, etc.

                    \item \mettype{OTHERS}

                        \emph{kinsen} (money) \emoji{u1f4b0},
                        \emph{jikan} (time) \emoji{u23f0},
                        %\emph{roudou} (labor) \emoji{u1f469-200d-1f527},
                        %\emph{keeki} (cake) \emoji{u1f370},
                        %\emph{ari} (ant) \emoji{u1f41c},
                        %\emph{suimin} (sleep) \emoji{u1f634},
                        \emph{neko} (cat) \emoji{u1f63a}, etc.

                \end{itemize}

                \bigskip

                \msubsection{Verbal Phrases}

                \begin{itemize}
                    \item \mettype{PHYSICAL ACTION RELATED TO WATER}

                        \emph{X ga ahureru} (\emph{X} overflows),
%                        \emph{X ni hitasu} (\emph{Y} dips Z in \emph{X}),
%                        \emph{X de susugu} (\emph{Y} rinses \emph{Z} with \emph{X}),
                        \emph{X wo nomu} (somebody drinks \emph{X}),
%                        \emph{X wo kumitoru} (\emph{Y} scoops up \emph{X}),
                        etc.
                \end{itemize}

            \end{block}

        \end{column}
    \end{columns}

    \vspace{11mm}

    \begin{columns}[t]
        \begin{column}{.968\textwidth}
            \begin{alertblock}{METRICS}

                \begin{columns}[t]
                    \begin{column}{.3\textwidth}
                        \msubsection{Metaphoricity}

                        \textbf{Metaphoricity} measures how metaphorical an expression is.

                        \begin{table}
                            \centering\scriptsize
                            \begin{tabular}{rllc}
                                \toprule%
                                \header{Rank} & \header{Noun (\emph{X})} & \header{Verbal phrase} & \header{Score} \\
                                \midrule%
                                1 & \emph{kotoba} (word)     & \emph{X ga huttou-suru} (\emph{X} boils)          & 3.9 \\
                                2 & \emph{kanjou} (emotion)  & \emph{X ni oboreru} (\emph{Y} almost drowns in \emph{X}) & 3.8 \\
                                3 & \emph{zetubou} (despair) & \emph{X ga ahureru} (\emph{X} overflows)          & 3.7 \\
                                \multicolumn{4}{c}{$\vdots$} \\
                                1356 & \emph{mizu} (water) & \emph{X ga huttou-suru} (\emph{X} boils)      & 0.0 \\
                                1356 & \emph{mizu} (water) & \emph{X ga nagareru} (\emph{X} flows)         & 0.0 \\
                                1356 & \emph{mizu} (water) & \emph{X wo nomu} (\emph{Y} drinks \emph{X})          & 0.0 \\
                                \bottomrule%
                            \end{tabular}
                        \end{table}

                        High-ranked expressions often use nouns related to emotion
                        such as \emph{zetubou} (despair).

                        \begin{figure}
                            \centering
                            \includegraphics[width=0.92\textwidth]{metaphoricity-en}
                        \end{figure}

                    \end{column}

                    \begin{column}{.3\textwidth}
                        \msubsection{Novelty}

                        \textbf{Novelty} measures
                        how novel an expression looks or sounds.

                        \begin{table}[!t]
                            \centering\scriptsize
                            \begin{tabular}{rllc}
                                \toprule%
                                \header{Rank} & \header{Noun (\emph{X})} & \header{Verbal phrase} & \header{Score} \\
                                \midrule%
                                1 & \emph{ari} (ant)             & \emph{X ga simiru} (\emph{X} soaks into \emph{Y})    & 4.0 \\
                                1 & \emph{neko} (cat)            & \emph{X wo siboridasu} (\emph{Y} squeezes \emph{X})  & 4.0 \\
                                1 & \emph{neko} (cat)            & \emph{X ga nagarederu} (\emph{X} flows out)   & 4.0 \\
                                \multicolumn{4}{c}{$\vdots$} \\
                                1353 & \emph{mizu} (water)      & \emph{X ga nagareru} (\emph{X} flows)             & 0.0 \\
                                1353 & \emph{mizu} (water)      & \emph{X ga waku} (\emph{X} gushes out)            & 0.0 \\
                                1353 & \emph{mizu} (water)      & \emph{X wo nomu} (\emph{Y} drinks \emph{X})              & 0.0 \\
                                \bottomrule%
                            \end{tabular}
                        \end{table}

                        Expressions that use nouns for concrete objects were ranked high.

                        \begin{figure}
                            \centering
                            \includegraphics[width=0.92\textwidth]{novelty-en}
                        \end{figure}

                    \end{column}

                    \begin{column}{.3\textwidth}
                        \msubsection{Comprehensibility}

                        \textbf{Comprehensibility}
                        measures how easy it is to understand the meaning of expressions.

                        \begin{table}[!t]
                            \centering\scriptsize
                            \begin{tabular}{rllc}
                                \toprule%
                                \header{Rank} & \header{Noun (\emph{X})} & \header{Verbal phrase} & \header{Score} \\
                                \midrule%
                                1 & \emph{ai} (love)         & \emph{X ni oboreru} (\emph{Y} almost drowns in \emph{X}) & 4.0 \\
                                1 & \emph{kanjou} (emotion)  & \emph{X wo kumitoru} (\emph{Y} scoops up \emph{X})       & 4.0 \\
                                1 & \emph{mizu} (water)      & \emph{X de susugu} (\emph{Y} rinses \emph{Z} with \emph{X})     & 4.0 \\
                                \multicolumn{4}{c}{$\vdots$} \\
                                1356 & \emph{ari} (ant)             & \emph{X ga simiru} (\emph{X} soaks into \emph{Y})    & 0.0 \\
                                1356 & \emph{iwa} (rock)            & \emph{X wo susuru} (\emph{Y} sips \emph{X}) & 0.0 \\
                                1356 & \emph{zouo} (hatred)         & \emph{X de susugu} (\emph{Y} rinses \emph{Z} with \emph{X}) & 0.0 \\
                                \bottomrule%
                            \end{tabular}
                        \end{table}

                        This metric shows the opposite tendency
                        to the novelty.
                        The corr.\ coefficient was −0.92.

                        \begin{figure}
                            \centering
                            \includegraphics[width=0.92\textwidth]{comprehensibility-en}
                        \end{figure}

                    \end{column}
                \end{columns}
            \end{alertblock}
        \end{column}
    \end{columns}
%    \end{adjustwidth}
    \vspace{11mm}

    \begin{columns}[t]
        \begin{column}{.47\textwidth}

            \begin{block}{OVERALL EVALUATION}
                \msubsection{Overall Evaluation}

                Here we defined it as
                the average of the three metrics.

                \bigskip
                \bigskip

                \msubsection{Goodness}

                A volunteer chose one
                from each pair of high- and low-ranked expressions
                that made him more inclined to use (marked red in the table).

                \smallskip

                \begin{table}[t]
                    \centering\scriptsize
                    \begin{tabular}{lcl}
                        \toprule%
                        \header{High-ranked expression [rank in overall eval.]}
                        &
                        & \header{Low-ranked expression [rank in overall eval.]} \\
                        \midrule
                        \better{\emph{human wo nomu} (\emph{Y} drinks complaints)} [23]
                            &
                            & \emph{abura wo kumitoru} (\emph{Y} scoops up oil) [1087] \\
                        \better{\emph{ikari ga koboreru} (anger spills out)} [6]
                            &
                            & \emph{iwa ni oboreru} (\emph{Y} almost drowns in a rock) [1117] \\
                        \better{\emph{syuutisin ga tamaru} (shame gets collected)} [44]
                            &
                            & \emph{syuutisin wo sosogu} (\emph{Y} pours shame) [856] \\
                        \better{\emph{jouhou ga nigoru} (information gets cloudy)} [106]
                            &
                            & \emph{kuuki wo makitirasu} (\emph{Y} scatters the air) [212] \\
                        \better{\emph{kanasimi ga simiru} (sorrow soaks into \emph{Y})} [32]
                            &
                            & \emph{rikai ga nagareru} (understanding flows) [721] \\
                        \emph{tanosisa ga uzumaku} (enjoyment whirls) [81]
                            &
                            & \better{\tiny \emph{human ni tukaru} (\emph{Y} gets submerged in complaints)} [1241] \\
                        \emph{kotoba ga nizimu} (words ooze) [14]
                            &
                            & \better{\emph{kyouhu ga nagareru} (fear flows)} [307] \\
                        \better{\emph{kanjou wo sosogu} (\emph{Y} pours emotion)} [44]
                            &
                            & {\tiny \emph{ito X ni tukaru} (\emph{Y} gets submerged in intention) [654]} \\
                        \better{\emph{huan ga nagarederu} (anxiety flows out)} [44]
                            &
                           & \emph{jounetu wo kumitoru} (\emph{Y} scoops up passion) [165] \\
                        \better{\tiny \emph{jouhou ni oboreru} (\emph{Y} almost drowns in information)} [23]
                            &
                           & \emph{abura ga tamaru} (oil gets collected) [1241] \\
                        \bottomrule
                    \end{tabular}
                \end{table}

                \bigskip

                \msubsection{Metaphoricity}

                The author judged 8 of the 10 expressions as metaphorical
                by using MIPVU \citep{steen2010}.
            \end{block}

        \end{column}

        \begin{column}{.47\textwidth}

            \begin{block}{CONCLUSION}
                \setlength\leftmargini{2.8em}
                \begin{itemize}
                    \item We proposed metrics to evaluate automatically generated metaphors.

                    \item We actually evaluated expressions by crowdsourcing.

                    \item The result shows the validity of the metrics
                            and their relationship.

                    \item High-ranked expressions
                        in the overall eval.\ are good metaphors.
                \end{itemize}

                \bigskip
                \bigskip

                The results are available at
                \href{https://github.com/pecorarista/metaphor-evaluation-result}{\textcolor{sDarkBlue}{https://github.com/pecorarista/metaphor-evaluation-result}}.
            \end{block}

            \begin{block}{REFERENCE}
                \setbeamertemplate{bibliography item}[text]
                \begin{indentation}{1.3em}{1em}
                    \printbibliography
                \end{indentation}
            \end{block}
        \end{column}
    \end{columns}
\end{frame}
\end{document}
