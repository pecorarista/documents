\documentclass[10pt,hyperref={unicode}]{beamer}
\usepackage{lmodern}
\usepackage{amsmath,amssymb,accents,mathrsfs}
\usepackage{datetime}
\usepackage[noenc,safe]{tipa}
\usepackage{luatextra}
\usepackage{luatexja-otf}
\usepackage{luatexja-fontspec}
\usepackage[hiragino-pron,deluxe]{luatexja-preset}
\usepackage{polyglossia}
\setmainlanguage{english}
\setotherlanguage{russian}
\usepackage{tikz}
\usepackage{tikz-3dplot}
\usepackage{pgf}
\usepackage{pgfplots}
\usetikzlibrary{arrows,quotes,angles}
\usepackage{scrextend}
\usepackage{natbib}
\deffootnote[10pt]{10pt}{10pt}{\makebox[10pt][l]{\thefootnotemark\hspace{10pt}}}
\usetheme{default}
\usecolortheme{metropolis}
\usefonttheme{professionalfonts}
\setmainjfont[BoldFont=Hiragino Kaku Gothic ProN W6,Ligatures=TeX]%
{Hiragino Kaku Gothic ProN W3}
\setsansfont[BoldFont=Hiragino Kaku Gothic ProN W6,Ligatures=TeX]{Hiragino Kaku Gothic ProN W3}
%\setmonofont{DejaVu Sans Mono}
\newfontfamily\ipafont[]{CMU Sans Serif}
\DeclareMathOperator*{\aff}{aff}
\DeclareMathOperator*{\argmax}{arg\,max}
\DeclareMathOperator*{\epi}{epi}
\DeclareMathOperator*{\id}{id}
\DeclareMathOperator*{\rank}{rank}
\DeclareMathOperator*{\ri}{ri}
\DeclareMathOperator*{\sgn}{sgn}
\DeclareMathOperator*{\tr}{tr}
\DeclareMathOperator*{\var}{var}
\DeclareMathOperator*{\dom}{dom}
\newcommand\dottedcircle{%
\begin{pgfpicture}
\pgfsetlinewidth{0.25ex}
\pgfsetroundcap
\pgfsetdash{{0cm}{2pt}{0cm}{2pt}}{0cm}
\pgfcircle{\pgfpointorigin}{0.75ex}
\pgfusepath{stroke}
\pgfsetbaseline{-0.75ex}
\end{pgfpicture}%
}
\setbeamertemplate{navigation symbols}{}
\setbeamertemplate{itemize/enumerate subbody begin}{\vspace{1em}}
\setbeamertemplate{caption}{\raggedright\insertcaption\par}
\definecolor{mDarkTeal}{HTML}{23373b}
\setbeamercolor{block title}{use=structure,fg=white,bg=mDarkTeal}
\setbeamercolor{block body}{use=structure,fg=black,bg=gray!20!white}
\newenvironment{wideitemize}{\itemize\addtolength{\itemsep}{1em}}{\enditemize}
\newenvironment{wideenumerate}{\enumerate\addtolength{\itemsep}{1em}}{\endenumerate}
\addtobeamertemplate{navigation symbols}{}{%
\usebeamerfont{footline}%
\usebeamercolor[fg]{footline}%
\hspace{1em}%
\insertframenumber/\inserttotalframenumber
}
\makeatletter\def\tagform@#1{\maketag@@@{{\fontfamily{cmr}\selectfont(#1)}\@@italiccorr}}\makeatother
\newcommand{\braref}[1]{{\fontfamily{cmr}\selectfont (\ref{#1})}}
\makeatletter\renewenvironment{proof}{\par\pushQED{\qed}\normalfont\topsep6\p@\@plus6\p@\relax\trivlist\item[\hskip\labelsep{\bfseries 証明}\hskip\labelsep]\ignorespaces}{\popQED\endtrivlist\@endpefalse}\makeatother
\newcommand{\parentheses}[1]{\left(#1\right)}
\newcommand{\leftopenrightclose}[1]{\left(\left.#1\right]\right.}
\newcommand{\braces}[1]{\left\{#1\right\}}
\newcommand{\brackets}[1]{\left[#1\right]}
\newcommand{\anglebrackets}[1]{\left\langle#1\right\rangle}
\newcommand{\norm}[1]{\left\|#1\right\|}
\newcommand{\absolute}[1]{\left|#1\right|}
\renewcommand{\qedsymbol}{\rule{5pt}{10pt}}
\pgfplotsset{%
width=5cm,
compat=newest,
xlabel near ticks,
ylabel near ticks
}
\title{最急降下法}
\institute{総合研究大学院大学 博士前期1年\newline\newline\texttt{miyazawa-a@nii.ac.jp}}
\author{宮澤 彬}
\date{\today}
\begin{document}
\begin{frame}
\maketitle
\end{frame}

\begin{frame}
\frametitle{最急降下法}

関数の停留点(特に極小点)を,反復的な計算で求めるにはどうすればよいか.
接線の傾きが負である点から,$0$に近づく方向に移動していけばよさそうである.

\bigskip

\begin{center}
    \begin{tikzpicture}[scale=0.9]
        \draw (0,0) node[below left] {$O$};
        \draw[->] (-0.5,0) -- (7,0) node[below left] {$x$};
        \draw[->] (0,-1) -- (0,4) node[below left] {$y$};
        \draw[color=red] plot [domain=-0.25:5] (\x,{((\x-3)^2)/4 + 0.5)});
        \draw[color=blue] plot [domain=1.25:2.25] (\x,{(-0.625*(\x-1.75)+((1.75-3)^2)/4 + 0.5)});
        \draw[color=blue] plot [domain=0:1] (\x,{(-1.25*(\x-0.5)+((0.5-3)^2)/4 + 0.5)});
        \draw[color=blue] plot [domain=2.5:3.5] (\x,0.5);
        \draw (5.6,2) node {$y=f\parentheses{x}$};
        \draw (4.8,0.375) node {$\nabla f\parentheses{x_\infty} = 0$};
        \draw[dotted] (0.5,2.0625) -- (0.5,0) node[below] {$x_0$};
        \draw[dotted] (1.75,0.890625) -- (1.75,0) node[below] {$x_k$};
        \draw[dotted] (3,0.5) -- (3,0) node[below] {$x_\infty$};
    \end{tikzpicture}
\end{center}
\end{frame}

\begin{frame}
\frametitle{Armijo条件}
$0 < \xi_1 < 1$であるような定数$\xi_1$に対して,
\begin{align*}
    f\parentheses{x_k + \alpha d_k} \leq f \parentheses{x_k} + \xi_1 \alpha \nabla f\parentheses{x_k}\cdot d_k
\end{align*}
を満たす$\alpha > 0$を選ぶ.この条件を\textbf{Armijo条件}\footnote[frame]{スペイン語読みをするならばおそらく\begin{ipafont}\textipa{/ar\textprimstress mixo/}\end{ipafont}.}という.

\smallskip

\begin{center}
    \begin{tikzpicture}[scale=0.9]
        \draw (0,0) node[below left] {$O$};
        \draw[->] (-0.5,0) -- (7,0) node[below left] {$x$};
        \draw[->] (0,-1.5) -- (0,4) node[below left] {$y$};
        \draw[color=red] plot [domain=-0.25:5] (\x,{((\x-3)^2)/4 + 0.5)});
        \draw[color=red,thick] plot [domain=0.5:3.5] (\x,{((\x-3)^2)/4 + 0.5)});
        \draw[color=blue] plot [domain=-0.25:3] (\x,{(-1.25*(\x-0.5)+((0.5-3)^2)/4 + 0.5)});
        \draw[color=blue] plot [domain=-0.25:4] (\x,{(-0.5*(\x-0.5)+((0.5-3)^2)/4 + 0.5)});
        \draw (6.8,0.375) node {$y = f\parentheses{x_k} + \xi_1 \alpha \nabla f\parentheses{x_k}\cdot d_k$};
        \draw (5.6,-1) node {$y = f\parentheses{x_k} + \alpha \nabla f\parentheses{x_k}\cdot d_k$};
        \draw (5.6,2) node {$y=f\parentheses{x}$};
        \draw[dotted] (0.5,2.0625) -- (0.5,0) node[below] {$x_k$};
        \draw[dotted] (3.5,0.5625) -- (3.5,0);
        \draw[thick] (0.5,0) -- (3.5,0) node[below] {$x_k + \alpha d_k$};
    \end{tikzpicture}
\end{center}


\end{frame}

\begin{frame}
\frametitle{Wolfe条件}
$0 < \xi_1 < \xi_2 < 1$であるような$\xi_1,\,\xi_2$に対して
\begin{align*}
    \xi_2 \nabla f\parentheses{x_k}\cdot d_k \leq \nabla f \parentheses{x_k + \alpha d_k}\cdot d_k
\end{align*}
を満たす$\alpha > 0$を選ぶ.この条件を\textbf{曲率条件}(curvature condition)と呼ぶ.この条件とArmijo条件を合わせて\textbf{Wolfe条件}と呼ぶ.

\smallskip

\begin{center}
    \begin{tikzpicture}[scale=0.9]
        \draw (0,0) node[below left] {$O$};
        \draw[->] (-0.5,0) -- (7,0) node[below left] {$x$};
        \draw[->] (0,-1) -- (0,4) node[below left] {$y$};
        \draw[color=red] plot [domain=-0.25:5] (\x,{((\x-3)^2)/4 + 0.5)});
        \draw[color=blue] plot [domain=-0.5:1.5] (\x,{(-1.25*(\x-0.5)+((0.5-3)^2)/4 + 0.5)});
        \draw[color=blue] plot [domain=-0.5:1.5] (\x,{(-0.5*(\x-0.5)+((0.5-3)^2)/4 + 0.5)});
        \draw[color=blue] plot [domain=1:3] (\x,{(-0.5*(\x-2)+((2-3)^2)/4 + 0.5)});
        \draw (2.5,1.6) node {$\xi_2 \nabla f\parentheses{x_k}$};
        \draw (1.27,0.45) node {$\nabla f\parentheses{x_k}$};
        \draw (5.6,2) node {$y=f\parentheses{x}$};
        \draw[dotted] (0.5,2.0625) -- (0.5,0) node[below] {$x_k$};
        \draw[dotted] (2,0.75) -- (2,0) node[below] {$x_k + \alpha d_k$};
        \draw[thick] (2,0) -- (7,0);
    \end{tikzpicture}
\end{center}
\end{frame}

\begin{frame}
\frametitle{Zoutendijk条件}
\textbf{定理}{\hskip\labelsep}目的関数$f\parentheses{x}$は下に有界で,
かつ,初期点$x_0$における準位集合
$\braces{x\;;\; f\parentheses{x} \leq f\parentheses{x_0}}$における
を含む開集合$U$において連続的微分可能であるとする.
また勾配$\nabla f\parentheses{x}$は$U$でLipschitz連続であるとする.
すなわち,ある正定数$L$が存在して,任意の$x,\,y \in U$に対して
\begin{align*}
    \norm{\nabla f\parentheses{x} - \nabla f\parentheses{y}} \leq L \norm{x - y}
\end{align*}
が成り立つとする.

\bigskip

このとき$x_{k + 1} = x_k + \alpha_k d_k$を以下の条件を満たすようにとる.
\begin{itemize}
    \item 各$\alpha_k$がWolfe条件を満たす.
    \item 各$d_k$が降下方向である.すなわち$\nabla f\parentheses{x_k} \cdot d_k < 0$を満たす.
\end{itemize}
すると点列$\parentheses{x_k}_k$について
\begin{align*}
    \sum_{k = 0}^\infty \parentheses{\frac{\nabla f \parentheses{x_k}\cdot d_k}{\norm{d_k}}}^2 < \infty
\end{align*}
が成り立つ.
\end{frame}

\begin{frame}
\frametitle{Zoutendijk条件}
\textbf{証明}{\hskip\labelsep}曲率条件と$x_{k + 1} = x_k + \alpha_k d_k$から
\begin{gather*}
    \xi_2 \nabla f\parentheses{x_k}\cdot d_k \leq \nabla f \parentheses{x_{k + 1}}\cdot d_k\\
    \parentheses{\xi_2 - 1} \nabla f\parentheses{x_k}\cdot d_k \leq \parentheses{\nabla f \parentheses{x_{k + 1}} - \nabla f\parentheses{x_k}}\cdot d_k
\end{gather*}
が成り立つ.
Lipschitz条件より
\begin{align*}
    \parentheses{\nabla f \parentheses{x_{k + 1}} - \nabla f\parentheses{x_k}}\cdot d_k
    &\leq \left\|\nabla f \parentheses{x_{k + 1}} - \nabla f\parentheses{x_k}\right\|\left\| d_k\right\| \\
    &\leq L\left\| x_{k + 1} -x_k \right\|\left\| d_k\right\| \\
    &\leq \alpha_k L\left\|d_k\right\|^2
\end{align*}
が成り立つ.これらから
\begin{align*}
    \alpha_k
    &\geq \frac{\parentheses{\nabla f \parentheses{x_{k + 1}} - \nabla f\parentheses{x_k}}\cdot d_k}{L\left\|d_k\right\|^2} \\
    &\geq \frac{\xi_2 - 1}{L}\frac{\nabla f\parentheses{x_k}\cdot d_k}{\left\|d_k\right\|^2}
\end{align*}
を得る.
\end{frame}

\begin{frame}
\frametitle{Zoutendijk条件}
得られた$\alpha_k$をArmijo条件に代入して
\begin{align*}
    f\parentheses{x_{k + 1}} &\leq f \parentheses{x_k} + \xi_1 \alpha_k \nabla f\parentheses{x_k}\cdot d_k \\
                             &\leq f \parentheses{x_k} - \frac{\xi_1\parentheses{1 - \xi_2}}{L}\frac{\parentheses{\nabla f\parentheses{x_k}\cdot d_k}^2}{\left\|d_k\right\|^2}
\end{align*}
となる.ここで$k = 0$から$m$までの和をとると
\begin{gather*}
    \sum_{k = 0}^m \parentheses{f\parentheses{x_{k + 1}} - f \parentheses{x_k}}
    \leq - \sum_{k = 0}^m \frac{\xi_1\parentheses{1 - \xi_2}}{L} \frac{\parentheses{\nabla f\parentheses{x_k}\cdot d_k}^2}{\left\|d_k\right\|^2} \\
    f\parentheses{x_{m + 1}} - f \parentheses{x_0}
    \leq - \frac{\xi_1\parentheses{1 - \xi_2}}{L}\sum_{k = 0}^m \frac{\parentheses{\nabla f\parentheses{x_k}\cdot d_k}^2}{\left\|d_k\right\|^2}
\end{gather*}
を得る.
\end{frame}

\begin{frame}
\frametitle{Zoutendijk条件}
上式の右辺は$m$が増加するにつれて単調に減少する.また$f$は下に有界であると仮定していたので
\begin{align}
    \sum_{k = 0}^\infty \frac{\parentheses{\nabla f \parentheses{x_k} \cdot d_k}^2}{\left\|d_k\right\|^2} < \infty \tag{Zoutendijk} \label{Zoutendijk}
\end{align}
を得る.\hfill \rule{5pt}{10pt}

\bigskip

上の\braref{Zoutendijk}を\textbf{Zoutendijk条件}\footnote[frame]{オランダ語読みをするならばおそらく\begin{ipafont}\textipa{/\textprimstress zAut@nd\textlowering{E}Ik/}\end{ipafont}.}と呼ぶ.
\end{frame}


\begin{frame}
\frametitle{Zoutendijk条件}
Zoutendijk条件が成り立つとする.このとき
$S := \sum_{k = 0}^\infty \parentheses{\nabla f \parentheses{x_k} \cdot d_k}^2/\norm{d_k}^2$は
ある有限の値である.Cauchy-Schwarzの不等式から,任意の自然数$m$について
\begin{align*}
    \parentheses{\sum_{k = 0}^m \frac{\absolute{\nabla f \parentheses{x_k} \cdot d_k}}{\left\|d_k\right\|}}^2 \leq
    \sum_{k = 0}^m \frac{\parentheses{\nabla f \parentheses{x_k} \cdot d_k}^2}{\left\|d_k\right\|^2} \leq S
\end{align*}
が成り立つ.ゆえに
\begin{align*}
    \sum_{k = 0}^\infty \frac{\absolute{\nabla f \parentheses{x_k} \cdot d_k}}{\left\|d_k\right\|} \leq \sqrt{S}
\end{align*}
となり,この級数は収束することが分かる.したがって
\begin{align*}
    \frac{\absolute{\nabla f \parentheses{x_k} \cdot d_k}}{\left\|d_k\right\|} \to 0 \ \parentheses{k \to \infty}
\end{align*}
となる.
\end{frame}

\begin{frame}
\frametitle{最急降下法の大域収束性}
特に$d_k = - \nabla f\parentheses{x_k}$をとる.
この$d_k$は$\nabla f\parentheses{x_k} \cdot d_k = - \norm{f\parentheses{x_k}}^2 < 0$を満たすので,
降下方向である.さらに先に示した結果から,
\begin{align*}
    \frac{\absolute{\nabla f \parentheses{x_k} \cdot d_k}}{\norm{d_k}} = \norm{f\parentheses{x_k}}  \to 0 \ \parentheses{k \to \infty}
\end{align*}
を満たす.

\bigskip

Cauchy-Schwarzの不等式における等号成立条件から,
$\norm{d_k}$を固定して考えたとき,
この$d_k$は$\nabla f\parentheses{x_k} \cdot d_k$
を最小にするものである.
つまり最も急に減少させるものである.
そのため$d_k = - \nabla f\parentheses{x_k}$とする
方法を\textbf{最急降下法}(steepest descent method)と呼ぶ.
\end{frame}

%\bibliographystyle{apalike}
\begin{frame}
\frametitle{参考文献・おわりに}
主に以下を参考にした.

\bigskip

\begin{itemize}
    \item 矢部博, 新・工科系の数学「工学基礎 最適化とその応用」,
        数理工学社, 2006.
\end{itemize}

\bigskip

また,このスライドのソースコードは
\href{https://github.com/pecorarista/documents}{https://github.com/pecorarista/documents}にある.
%@book{yabe2006,
%  title={工学基礎最適化とその応用},
%  series={新・工科系の数学},
%  author={矢部博},
%  year={2006},
%  publisher={数理工学社}
%}
\end{frame}
\end{document}
