\documentclass[unicode,colorlinks]{beamer}
\usepackage[orientation=landscape,size=a0,scale=1.4]{beamerposter}
\usetheme[footertext={フッターに何か適当なことを書くこともできます.}]{SimplePoster}
\usepackage{amsmath,amssymb}
\usepackage{listings}
\usepackage[safe]{tipa}
\usepackage[no-math]{fontspec}
\usepackage{luatexja}
\usepackage{luatexja-otf}
\usepackage[match]{luatexja-fontspec}
\usepackage[hiragino-pron,deluxe,expert]{luatexja-preset}
\usepackage{booktabs}
\usepackage{tabularx}
\usepackage{tikz}
\usetikzlibrary{arrows.meta}
\usepackage{pgfplots}
\usetikzlibrary{arrows,quotes,angles}
\setmainfont[Ligatures=TeX]{CMU Serif}
\setsansfont[Ligatures=TeX]{Droid Sans}
\setmonofont{DejaVu Sans Mono}
\newcommand{\absolute}[1]{\left|#1\right|}
\newcommand{\parentheses}[1]{\left(#1\right)}
\newcommand{\sequence}[1]{\parentheses{#1}}
\renewcommand{\kanjifamilydefault}{\gtdefault}
\usefonttheme{professionalfonts}
% http://tex.stackexchange.com/questions/64830/using-tipa-with-fontspec
% http://tex.stackexchange.com/questions/264452/ipa-characters-with-tipa-in-beamer-poster-switch-to-small-font-size-cannot-be-s
\newfontfamily\ipafamily{Charis SIL}[Scale=MatchLowercase]
\DeclareRobustCommand\ipa[1]{{\ipafamily\textipa{#1}}}
\definecolor{royalblue}{RGB}{1,72,152}
\definecolor{skyblue}{RGB}{108,166,205}
\ltjsetparameter{jacharrange={-2,-3}}
\renewcommand{\baselinestretch}{1.2}
\newcommand\simpletheorem[2]{\rmfamily \textbf{#1}\; #2}
\lstset{%
    language={Python},
    basicstyle={\ttfamily},%
    keywordstyle={\color{red}\ttfamily\bfseries},%
    commentstyle={\color{gray}\ttfamily},
    stringstyle={\ttfamily},
    xleftmargin=2em,
}
\pgfplotsset{%
    compat=newest,
    xlabel near ticks,
    ylabel near ticks
}
\makeatletter
\renewcommand\hrulefill{\leavevmode\leaders\hrule height 4pt\hfill\kern\z@}
\makeatother
\setlength\heavyrulewidth{3pt}
\newcolumntype{C}{>{\centering\arraybackslash}X}
\hypersetup{%
    pdfinfo={%
        CreationDate={D:20160306040000},
        ModDate={D:20160306040000}
    }
}
\title{ポスターのサンプル}
\author{著者名}
\institute{所属機関名}
\date{\today}
\begin{document}
% fragile は listings を使うために必要
% http://tex.stackexchange.com/questions/130109/cant-insert-code-in-my-beamer-slide
\begin{frame}[fragile]
\begin{columns}[t]
    \begin{column}{.32\linewidth}
        \begin{block}{はじめに}
            \LaTeX{} (Beamer)でポスターを作りましょう.
        \end{block}
        \begin{block}{目次}
            過去に作った\LaTeX ファイルがそのまま流用できます.

            \bigskip
            \bigskip

            \begin{center}
                \begin{tabularx}{0.5\textwidth}{CC}
                    \toprule
                             & スコア \\
                    \midrule
                    手法1    & 0.11   \\
                    手法2    & 0.22   \\
                    手法3    & 0.33   \\
                    手法2    & 0.44   \\
                    \bottomrule
                \end{tabularx}
            \end{center}
        \end{block}

        \begin{block}{数式}
            \LaTeX{} で作るので数式の挿入が簡単にできます.

            \bigskip
            \bigskip

            \simpletheorem{Lebesgueの収束定理}{%
                $\sequence{f_n}$は $X$ 上可測関数列で,ある非負可積分関数 $g$ により
                \begin{align*}
                    \absolute{f_n\parentheses{x}} \leq g\parentheses{x} \quad \parentheses{x \in X}
                \end{align*}
                とする.極限 ${\displaystyle \lim_{n \to \infty}f_n\parentheses{x} = f\parentheses{x}}$ が存在するなら
                \begin{align*}
                    \int_X \lim_{n \to \infty}f\parentheses{x} \mu\parentheses{dx}
                    = \lim_{n \to \infty}\int_X f_n \parentheses{x} \mu\parentheses{dx}
                \end{align*}
            }
        \end{block}

        必ず\texttt{beamercolorbox}の中に書かなければならない,というわけではありません.
    \end{column}
    \begin{column}{.32\linewidth}
        \begin{block}{グラフ}
            TikZでさくっとグラフを描きましょう.

            \bigskip

            \tikzset{%
                declare function={%
                    normcdf(\x,\m,\s)=1/(1 + exp(-0.07056*((\x-\m)/\s)^3 - 1.5976*(\x-\m)/\s));
                },
                % http://tex.stackexchange.com/questions/5461/is-it-possible-to-change-the-size-of-an-arrowhead-in-tikz-pgf
                % TikZ manual p. 202
                >={Straight Barb[width=5mm,length=5mm]}
            }
            \begin{center}
                \begin{tikzpicture}
                    \begin{axis}[
                        samples=500,
                        domain=-10:10,
                        width=30cm, height=20cm,
                        xmin=-10.5, xmax=10.5,
                        ymin=-0.3, ymax=1.5,
                        axis x line=center,
                        axis y line=center,
                        xlabel={$x$},
                        ylabel={$y$},
                        axis line style={->, ultra thick},
                        xtick={-5,0,5},
                        ytick={0,0.5}]
                    \addplot[color=red,ultra thick] plot{normcdf(0.626657*x,0,1)} node[pos=0.4](cum){};
                    \addplot[color=blue,ultra thick,dashed] plot (\x,{1/(1+exp((-1)*x))}) node[pos=0.6](a){} ;
                    \node [color=blue,below right] at (a) {${\displaystyle y=\frac{1}{1+\exp(-x)}}$};
                    \node [color=red,above left] at (cum) {${\displaystyle y=\varPhi\left(\sqrt{\frac{\pi}{8}}x\right)}$};
                    \addplot[dashed,ultra thick] plot (\x,{1}) node[pos=0.5](b){};
                    \node [above left] at (b) {$y=1$};
                    \node at (0,0) [anchor=north west]{$O$};
                    \end{axis}
                \end{tikzpicture}
            \end{center}
        \end{block}

        \begin{block}{プログラム}
            \texttt{listings}パッケージを読み込んでプログラムのソースコードを書くことができます.
            フレームに\texttt{fragile}オプションを渡す必要があるので気をつけましょう.

            \bigskip

            \hrulefill

            \begin{lstlisting}
# This function does not always work!
def plural(word):
    if word.endswith('y'):
        return word[:-1] + 'ies'
    elif word[-1] in 'sx':
        return word + 'es'
    else:
        return word + 's'
            \end{lstlisting}
        \end{block}
    \end{column}
    \begin{column}{.32\linewidth}
        \begin{block}{諸文字}
            キリル文字やギリシャ文字なども自由に入れられます.

            \bigskip
            \bigskip

            \begin{itemize}
                \item いろはにほへとちりぬるを
                    わかよたれそつねならむ
                    うゐのおくやまけふこえて
                    あさきゆめみしゑひもせす
                \item The quick brown fox jumps over the lazy dog.
                \item Falsches Üben von Xylophonmusik quält jeden größeren Zwerg.
                \item Широкая электрификация южных губерний даст
                    мощный толчок подъёму сельского хозяйства.
                \item Ταχίστη αλώπηξ βαφής ψημένη γη, δρασκελίζει υπέρ νωθρού κυνός.
            \end{itemize}
        \end{block}

        \begin{block}{発音記号}
            \texttt{tipa}パッケージを使えば,複雑な発音記号の出力も簡単です.

            \bigskip
            \bigskip

            \begingroup
            \large
            \ipa{
                u \textupstep{}"v\~{e}tu
                "nORt
                \textupstep{}i u
                sOl \textsubring{d}Sku"ti""\~{5}u
                \textupstep{}k\textsubarch{u}al
                duZ doiz
                \textupstep{}"ERO maiS
                \tone{13}fORtW
                \textpipe\,\textupstep{}"k\textsubarch{u}\~{5}du
                susW"deu p5"saR \~{u} vi5\textupstep"Z\~{5}tW
                \textsubring{\~{W}}\textupstep{}"volt num5 \tone{31}kap5
                }
            \endgroup
        \end{block}

        \begin{block}{参考文献}
            \begin{enumerate}
                \item 小谷眞一「測度と確率」, 岩波書店 (2005).
                \item 国際音声学会編, 竹林滋・神山孝夫訳「国際音声記号ガイドブッ

                    ク」, 大修館書店 (2003).
            \end{enumerate}
        \end{block}

        \href{https://github.com/deselaers/latex-beamerposter}{\texttt{beamerposter}パッケージ}
        の\texttt{examples}を参考にして自分用の\texttt{.sty}ファイルを作成しましょう.
    \end{column}
\end{columns}
\end{frame}
\end{document}
