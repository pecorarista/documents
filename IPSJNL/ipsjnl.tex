\documentclass[12pt,usepdftitle=false]{beamer}
\usepackage[sort]{natbib}
\usepackage{tikz}
\usepackage{colortbl}
\usepackage{ccicons}
\usepackage{booktabs}
\usepackage{framed}
\usepackage{amssymb}
\usepackage{luatexja-ruby}
\usepackage[no-math]{fontspec}
\newfontfamily\phonetic{Times LT Std Phonetic}
\ltjsetparameter{%
    jacharrange={%
        -2, % Exclude Greek and Cyrillic letters.
        -3  % Punctuations and Miscellaneous symbols.
    },
    alxspmode={`#,allow},
    alxspmode={`/,allow}
}
\usetheme{sakura}
\usepackage[bottom,flushmargin,hang,multiple]{footmisc}
\newcommand\emoji[1]{\hspace{.25\zw}\raisebox{-2pt}{\includegraphics[height=11pt]{#1.pdf}}\hspace{.25\zw}}
\newcommand\header[1]{\multicolumn{1}{c}{\textbf{#1}}}
\newcommand\tightlist{\itemsep1pt\parskip0pt\parsep0pt}
\definecolor{sDarkBlue}{HTML}{3B55CD}
\setbeamerfont{subtitle}{size=\LARGE,series=\bfseries}
\title{メタファーの自動生成に向けた}
\subtitle{客観的評価指標の検討}
\institute{総合研究大学院大学/国立情報学研究所}
\author{\underline{宮澤 彬}\hspace{3em}宮尾 祐介}
\date{2017年5月16日}
\hypersetup{%
    unicode,
    colorlinks,
    allcolors=sDarkBlue,
    pdfinfo={%
        Title={メタファーの自動生成に向けた客観的評価指標の検討},
        Author={宮澤 彬}
    }
}
\usepackage{gb4e,cgloss4e}
\noautomath%
\let\eachwordtwo=\sffamily
\let\eachwordthree=\sffamily

% https://tex.stackexchange.com/questions/24136/natbib-and-hyperref-for-author-year-style-produces-two-links
\usepackage{etoolbox}
\makeatletter

\pretocmd{\NAT@citex}{%
  \let\NAT@hyper@\NAT@hyper@citex
  \def\NAT@postnote{#2}%
  \setcounter{NAT@total@cites}{0}%
  \setcounter{NAT@count@cites}{0}%
  \forcsvlist{\stepcounter{NAT@total@cites}\@gobble}{#3}}{}{}
\newcounter{NAT@total@cites}
\newcounter{NAT@count@cites}
\def\NAT@postnote{}

% include postnote and \citet closing bracket in hyperlink
\def\NAT@hyper@citex#1{%
  \stepcounter{NAT@count@cites}%
  \hyper@natlinkstart{\@citeb\@extra@b@citeb}#1%
  \ifnumequal{\value{NAT@count@cites}}{\value{NAT@total@cites}}
    {\ifNAT@swa\else\if*\NAT@postnote*\else%
     \NAT@cmt\NAT@postnote\global\def\NAT@postnote{}\fi\fi}{}%
  \ifNAT@swa\else\if\relax\NAT@date\relax
  \else\NAT@@close\global\let\NAT@nm\@empty\fi\fi% avoid compact citations
  \hyper@natlinkend}
\renewcommand\hyper@natlinkbreak[2]{#1}

% avoid extraneous postnotes, closing brackets
\patchcmd{\NAT@citex}
  {\ifNAT@swa\else\if*#2*\else\NAT@cmt#2\fi
   \if\relax\NAT@date\relax\else\NAT@@close\fi\fi}{}{}{}
\patchcmd{\NAT@citex}
  {\if\relax\NAT@date\relax\NAT@def@citea\else\NAT@def@citea@close\fi}
  {\if\relax\NAT@date\relax\NAT@def@citea\else\NAT@def@citea@space\fi}{}{}

\makeatother
\hypersetup{%
    unicode,
    colorlinks,
    allcolors=sDarkBlue,
    pdfinfo={%
        Title={メタファーの自動生成に向けた客観的評価指標の検討},
        Author={宮澤 彬},
        Keywords={自然言語処理, メタファー, 生成}
    }
}
\begin{document}

\begin{frame}
    \maketitle
\end{frame}


\section{はじめに}
\begin{frame}
\frametitle{日常表現に溢れるメタファー}
\textbf{メタファー}は感情など抽象的なものを理解する上で不可欠である.
例えば以下のような表現が慣用的に使われている.

    \begin{itemize}
        \item 愛情を注ぐ \emoji{love}\emoji{needle}
%        \item 欲望に溺れる \emoji{love}\emoji{needle}
        \item 優越感に\ruby{浸}{ひた}る \emoji{hokkori}\emoji{bathing}
        \item 不満を漏らす \emoji{confused}\emoji{drop}
        \item 勇気が湧く \emoji{boast}\emoji{fountain}
        \item 怒りに油を注ぐ \emoji{angry}\emoji{gasoline}
        \item 憎悪の炎 \emoji{grimace}\emoji{fire}
        \item 爆発する \emoji{explosion}
    \end{itemize}
\end{frame}

\begin{frame}
\frametitle{メタファー生成の必要性}
    既に定形表現になっているものも多いが,
    メタファーは新たな語義を生みだす創造的な言語活動である.

    (例)「炎上」\emoji{fire}の新しい用法

    \bigskip

    メタファー的な表現は,非メタファー的な表現と比較して,
    感情に及ぼす影響が大きい\citep{mohammad2016}.

    \bigskip

    政治的な発言の説得力を高める手段にもなる\citep{black2011}.
    \bigskip

    詩や小説の執筆支援や,演説の原稿作成の支援などの応用が考えられる.
\end{frame}

\begin{frame}
    \frametitle{本研究の動機と概要}

    \textbf{動機}

    メタファー生成では直喩(「\emph{S}のような\emph{T}」など)が中心に研究されてきた.
    一般のメタファーの生成ではどのようなメタファーを生成すべきなのか?

    (例)「?憎悪を注ぐ」

    \bigskip

    \textbf{概要}

    本研究では生成すべき「よいメタファー」が満たすべき性質(評価指標)を提案する.

    \bigskip

    クラウドソーシングを用いた実験により,
    提案する指標が評価可能であること,
    そして「よいメタファー」の発見に有効であることを示す.

\end{frame}

\section{先行研究のタスク設定}
\begin{frame}
    \frametitle{既存のメタファー生成タスク}

    \cite{jakitada2001}
    \addtolength{\leftmargini}{-1\zw}
    \begin{itemize}
        \item 入力:「\emph{X}は\emph{Y}」(\emph{Y}は動詞または形容詞)という形式の文

            (例)「彼の心は澄んでいる」

        \item 出力:「\emph{Z}のように」や「\emph{Z}のようだ」を挿入した文

            (例)「彼の心は\textbf{青い海のように}澄んでいる」

    \end{itemize}

    \bigskip

    \citet{abe2006}

    \begin{itemize}

        \item 入力:``\emph{S} like \emph{T}''の\emph{S}と,\emph{T}が持つ性質

            (例)\emph{S} = ``the character'',\newline
             \hspace{2.5\zw}\emph{T}: ``young, innocent and fine character''

        \item 出力:\emph{T}の候補

            (例)``puppy''や``cat''

    \end{itemize}
\end{frame}


\begin{frame}
    \frametitle{先行研究における評価指標}
    \citet{jakitada2001}は「メタファーとしてのよさ」を人手で評価している.
    また生成過程で以下を考慮している.
    \begin{enumerate}
        \item 主題との共起度(理解のしやすさに影響)
        \item 主題とのカテゴリ間類似度$\text{\emph{R}}_\text{\emph{C}}$(斬新さに影響)

            (例)$\text{\emph{R}}_\text{\emph{C}}$ (``心'', ``頭脳'') > $\text{\emph{R}}_\text{\emph{C}}$ (``心'', ``水'')

        \item 主題との情緒的類似度$\text{\emph{R}}_\text{\emph{S}}$(理解のしやすさに影響)

            (例)$\text{\emph{R}}_\text{\emph{S}}$ (``心'', ``頭脳'') < $\text{\emph{R}}_\text{\emph{S}}$ (``心'', ``水'')
    \end{enumerate}

    \bigskip

    \citet{abe2006}は以下の4点を人手で評価している.
    \begin{enumerate}
        \item 適切さ(adequacy)

        \item 映像化の容易性(ease of visualization)

        \item おもしろさ(amusingness)

        \item 新規性(novelty)

    \end{enumerate}
\end{frame}

\begin{frame}
    \frametitle{メタファーの定義}

    認知言語学では概念間の写像として定義されることが多い\citep{lakoff1999}.

    \bigskip

    例えば「快楽に溺れる」や「勇気が湧く」は,
    水に関連した概念を用いて感情について記述している.
    これらの背後にはという「水」の概念から「感情」の概念への写像があると考える.

    \bigskip

     \begin{figure}[b]
         \begin{tikzpicture}[line width=1pt]
             \draw [color=sDarkBlue] (0,0) circle (25pt);
             \node (A) at (0,0) {\textsc{\textbf{水}}};
             \node at (0,-1.2) {\small 根源領域};
             \draw [color=sDarkBlue] (4,0) circle (25pt);
             \node (B) at (4,0) {\textsc{\textbf{感情}}};
             \node at (4,-1.2) {\small 対象領域};
             \draw [->,color=sDarkBlue] (A) edge [bend left] (B);
         \end{tikzpicture}
         \caption{メタファーを写像として捉えた場合の図.}
     \end{figure}

\end{frame}

\begin{frame}[plain,noframenumbering]
    \frametitle{メタファー性の判断に関する先行研究}

    先の定義は,単にメタファーかどうかの判断をするためには不向きである.
    その目的のためには\citet{steen2010}のガイドライン(MIPVU)が利用できる.

	\bigskip

    \textbf{MIPVUの手順の概要}

    \begin{enumerate}
        \item \textbf{文脈語義}を特定する.
        \item 文脈語義より\textbf{基礎的な語義}\footnote[frame]{具体的であったり,想像したり見たり触ったりすることが容易なものを指す語義.}が存在するか確認する.
        \item それらが十分に異なるか判断する.
        \item それらに\textbf{ある種の類似性}が認められるか検討する.
    \end{enumerate}
\end{frame}

\begin{frame}
    \frametitle{MIPVUの詳細}
    例えば以下の文中の``struck''にアノテーションを行うとする.
    \begin{center}
        % VUAMC l. 215054
        His speech \textbf{struck} me as the feeblest of the day.
    \end{center}

    % strike ~ as 句動詞だがタグが PRP なので別々に分析
    \begin{leftbar}
        \textbf{strike} {\phonetic /straɪk/} \textsc{verb}
        \begin{enumerate}
            \item to hit against, or to crash into, someone or something
                \onslide<3->{← 文脈語義より基礎的な語義}
            \item to make someone have a particular opinion or feeling
                \onslide<2->{← 文脈語義}
        \end{enumerate}

         \hfill\textcolor{gray}{\emph{Macmillan Dictionary}}
    \end{leftbar}

	\onslide<4-> この``struck''はメタファー(metaphor-related word)である.
\end{frame}


\begin{frame}
    \frametitle{メタファー写像の妥当性を検証する研究}

	\citet{kuroda2005}や\citet{nabeshima2011}は
	写像モデルの体系性や生産性について検証するため
    根源領域の語と対象領域の語を組み合わせ
	その容認度を求めている.

    \begin{table}
        \footnotesize
        \setlength\extrarowheight{2pt}
        \begin{tabular}{|c|c|c|c|c|c|}
            \hline
            & 水 & 感情 & 気持ち & 不満 & 勇気 \\
            \hline
            \emph{X}が溢れる & \cellcolor{sDarkRed}\textcolor{white}{4.0} & \cellcolor{sDarkRed}\textcolor{white}{3.7} & \cellcolor{sDarkRed}\textcolor{white}{3.8} & \cellcolor{sDarkRed}\textcolor{white}{3.8} & \cellcolor{sDarkRed}\textcolor{white}{3.3} \\
            \hline
            \emph{X}がこぼれる & \cellcolor{sDarkRed}\textcolor{white}{4.0} & \cellcolor{sOrange}\textcolor{white}{2.5} & \cellcolor{sOrange}\textcolor{white}{2.7} & \cellcolor{sDarkRed}\textcolor{white}{3.2} & \cellcolor{sBlonde}1.0 \\
            \hline
            \emph{X}を撒き散らす & \cellcolor{sDarkRed}\textcolor{white}{3.8} &
\cellcolor{sDarkRed}\textcolor{white}{3.2}
            & \cellcolor{sOrange}\textcolor{white}{2.8} & \cellcolor{sDarkRed}\textcolor{white}{3.5} & \cellcolor{sBlonde}1.2 \\
            \hline
            \emph{X}が漏れる & \cellcolor{sDarkRed}\textcolor{white}{4.0} & \cellcolor{sDarkRed}\textcolor{white}{3.0} & \cellcolor{sBlonde}{1.3} & \cellcolor{sDarkRed}\textcolor{white}{3.8} & \cellcolor{sBlonde}1.0 \\
            \hline
%            \emph{X}を注ぐ & \cellcolor{sDarkRed}\textcolor{white}{3.3} & \cellcolor{sOrange}\textcolor{white}{2.5} & \cellcolor{sOrange}\textcolor{white}{2.7} & \cellcolor{sBlonde}1.2 & \cellcolor{sBlonde}1.8 \\
%            \hline
            \emph{X}が濁る & \cellcolor{sDarkRed}\textcolor{white}{4.0} & \cellcolor{sOrange}\textcolor{white}{2.2} & \cellcolor{sOrange}\textcolor{white}{2.8} & 0.3 & 0.7 \\
            \hline
            \emph{X}が滴る & \cellcolor{sDarkRed}\textcolor{white}{3.3} & \cellcolor{sBlonde}1.3 & 0.7 & \cellcolor{sBlonde}1.3 & 0.5 \\
            \hline
            \emph{X}が滲む & \cellcolor{sBlonde}{1.8} & \cellcolor{sBlonde}1.2 & 0.5 & 0.8 & 0.3 \\
            \hline
        \end{tabular}
        \caption{\citet{nabeshima2011}が《感情は水》の妥当性の検証に用いた表の一部.}
    \end{table}
\end{frame}

% \begin{frame}
%     \frametitle{本研究で提案する評価指標と先行研究との関連}
%
%     \begin{enumerate}
%         \item \textbf{メタファー性}
%
%             直喩の生成においてはあまり重要視されてこなかった
%             ため本研究で導入する.
%
%         \item \textbf{理解可能性}
%
%             \citet{jakitada2001}の情緒的類似度,
%             \citet{abe2006}の適切さと映像化の容易性に対応する.
%
%         \item \textbf{新規性}
%
%             \citet{jakitada2001}の共起度とカテゴリ間差異,
%             \citet{abe2006}のおもしろさと新規性に対応する.
%
%         \item \textbf{総合評価}
%
%             本研究で新たに導入する.
%
%     \end{enumerate}
% \end{frame}

\section{提案指標}
\begin{frame}
    \frametitle{提案指標}
    本研究では以下の4つの指標を導入する.

     \begin{enumerate}
         \item \textbf{メタファー性}

         \item \textbf{理解可能性}

         \item \textbf{新規性}

         \item \textbf{総合評価}

     \end{enumerate}

     \bigskip

     重視する性質によって生成される表現を
     変更できるよう,
     複数の指標を用いる.

\end{frame}

\begin{frame}
    \frametitle{提案指標1 メタファー性}
    生成された表現がどの程度メタファーらしいか.

    (例)水を注ぐ→メタファー性が低い\newline
     \phantom{(例)}\hspace{0.5\zw}愛情を注ぐ→メタファー性が高い

     \bigskip

    直喩の生成においてはあまり重要視
    されてこなかったため本研究で導入する.

	\bigskip

	MIPVUのような厳密な方法では,
	訓練や言語学の知識が必要になりコストが高い.

	→今回はクラウドソーシングで評価を行う.

	\bigskip

	今後応用を行うにおいて
    自動化が必要な場合は
    メタファーの検出の手法(\citet{shutova2011}など)が利用できる.

\end{frame}

\begin{frame}
    \frametitle{提案指標2 新規性}
    生成された表現がどの程度新しいと感じられるか.

    (例)心を汲み取る→新規性が低い\newline
     \phantom{(例)}\hspace{0.5\zw}頭脳を汲み取る→新規性が高い

    \bigskip

    詩などの創作活動において,
    日常的な使用法からの逸脱は重要で,
    メタファーはそのための手段の1つである\citep{leech2014}.

    \bigskip

    \citet{jakitada2001}のカテゴリ間類似度,
    \citet{abe2006}のおもしろさと新規性に対応する.

    \bigskip

%    \begin{leftbar}
%        \ldots\ Antony would \ldots\ put a tongue
%
%        In Every wound of Caesar, that should move
%
%        The stones of Rome to rise and mutiny.
%
%        \hfill \textcolor{gray}{William Shakespeare's \emph{Julius Caesar}}
%    \end{leftbar}

    今回はどの程度新しいと感じるかを人手(クラウドソーシング)で評価する.
    将来的には頻度などを用いて自動で評価する.
\end{frame}

\begin{frame}
    \frametitle{提案指標3 理解可能性}
    生成された表現がどの程度理解しやすいか.

    (例)心を汲み取る→理解可能性が高い\newline
     \phantom{(例)}\hspace{0.5\zw}頭脳を汲み取る→理解可能性が低い

     \bigskip

     理解しにくい表現は
     書き手の意図が伝わらないため,
     執筆支援システムとして
     生成すべき表現とは異なる.

     \bigskip

    \citet{jakitada2001}の情緒的類似度,
    \citet{abe2006}の適切さと映像化の容易性に対応する.

    \bigskip

    表現がどの程度理解しやすいかを人手で評価する.

\end{frame}

\begin{frame}
    \frametitle{提案指標4 総合評価}

    重視すべき性質が決まっていない場合に
    どのようなメタファーを生成すべきか?

    \bigskip

    \textbf{総合評価}というものを導入する.
    今回は以下のように単純な和で定義する.
    \begin{align*}
        \text{総合評価}=\text{メタファー性}+\text{新規性}+\text{理解可能性}
    \end{align*}

\end{frame}

\section{実験}
\begin{frame}
    \frametitle{実験の概要}

    2つの実験を行う.

    \bigskip

    \textbf{実験1}

    クラウドソーシングを用いて,
    各指標について評価可能であることを検証する.

    \bigskip

    \textbf{実験2}

    総合評価で上位になった表現が
    「よいメタファー」になっていることを検証する.
    ボランティアの大学院生1名と筆者によって行う.

\end{frame}

\begin{frame}
    \frametitle{実験で使用した名詞の一覧}
    基本的には,
    \citet{nabeshima2011}が
    水のメタファー(特に《感情は水》)の
    生産性を検証するのに用いた語を使用した.
    ただし\citet{nabeshima2011}には
    具体的な感情を表す語が少なかったため,
    太字の28の名詞を追加し,計40の名詞を利用した.

    \begin{table}
        \footnotesize
        \begin{tabular}{lllll}
            \toprule
            水 & 油 & 砂 & 岩 & \textbf{泥} \\
            感情 & 気持ち & \textbf{意図} & \textbf{理解} & \textbf{睡眠} \\
            情報 & 言葉 & \textbf{声} & \textbf{音} & \textbf{光} \\
            金銭 & 空気 & \textbf{時間} & \textbf{労働} & \textbf{におい} \\
            勇気 & 不満 & \textbf{嫉妬} & \textbf{安心} & \textbf{不安} \\
            \textbf{羞恥心}  & \textbf{怒り} & \textbf{憎悪} & \textbf{快楽} & \textbf{喜び} \\
            \textbf{楽しさ} & \textbf{愛} & \textbf{情熱} & \textbf{希望} & \textbf{絶望} \\
            \textbf{悲しみ} & \textbf{恐怖} & \textbf{ネコ} & \textbf{アリ} & \textbf{ケーキ} \\
            \bottomrule
        \end{tabular}
        \caption{評価対象の表現を作成するために使用した名詞の一覧.}
    \end{table}
\end{frame}

\begin{frame}
    \frametitle{実験で使用した動詞句の一覧}
    動詞句は,
    太字の6つの動詞句を追加し,合計34の動詞句を利用した.

    \begin{table}
        \footnotesize
        \begin{tabular}{llll}
            \toprule
            \emph{X}が\ruby{溢}{あふ}れる
            & \emph{X}がこぼれる & \emph{X}を撒き散らす & \emph{X}が溜まる \\
            \emph{X}が漏れる & \emph{X}が満ちる & \emph{X}を搾り出す & \emph{X}が渦巻く\\
            \emph{X}が湧く & \emph{X}が流れる & \emph{X}に溺れる & \emph{X}をかける \\
            \emph{X}を注ぐ & \emph{X}に\ruby{浸}{ひた}る & \emph{X}を浴びせる & \emph{X}が濁る \\
            \emph{X}が淀む & \emph{X}を撒く & \emph{X}が澄む & \emph{X}がしみる \\
            \emph{X}に浸す & \emph{X}を垂らす & \emph{X}に漬かる & \emph{X}がほとばしる \\
            \emph{X}が滴る & \emph{X}に浮く & \emph{X}が滲む & \emph{X}ですすぐ \\
            \textbf{\emph{X}を飲む} & \textbf{\emph{X}を啜る} & \textbf{\emph{X}に沈む} & \textbf{\emph{X}を汲み取る} \\
            \textbf{\emph{X}が流れ出る} & \textbf{\emph{X}が沸騰する} & & \\
            \bottomrule
        \end{tabular}
        \caption{評価対象の表現を作成するために使用した動詞句の一覧.}
    \end{table}
\end{frame}

\begin{frame}
    \frametitle{実験1の設定}
    Yahoo!クラウドソーシングを利用して
    募集した作業者に,
    各指標について5段階で評価をしてもらった.

    \bigskip

    10名の回答の平均を\textbf{得点}として,分析に利用する.
    なお集計の際には\citet{nabeshima2011}と比較するために0から4に補正した.

    \bigskip

    \begin{figure}[b]
        \centering
        \includegraphics[width=0.7\textwidth]{screenshot.png}
        \caption{質問の例.}
    \end{figure}

\end{frame}

%\begin{frame}
%    \frametitle{実験1の設定II}
%    \addtolength{\leftmargini}{-1\zw}
%    \begin{itemize}
%        \item \textbf{メタファー性} 次の表現を比喩だと感じますか.
%
%                \begin{enumerate}
%                    \item[5.] 比喩だと感じる
%
%                    \item[1.] 比喩だとは感じない
%                \end{enumerate}
%
%        \smallskip
%
%        \item \textbf{理解可能性} 次の表現は理解しやすいですか.
%
%                \begin{enumerate}
%                    \item[5.] 問題なく理解できる
%
%                    \item[1.] まったく理解できない
%                \end{enumerate}
%
%        \smallskip
%
%        \item \textbf{新規性} 次の表現は新しい表現ですか.
%
%            \begin{enumerate}
%                \item[5.] 使ったこと,見聞きしたことがない新しい表現である
%
%                \item[1.] 広く使われている慣用的な表現である
%            \end{enumerate}
%
%    \end{itemize}
%\end{frame}
%



\begin{frame}
    \frametitle{メタファー性の実験結果I}
    \begin{figure}[h]\centering
        \vfill
        \includegraphics[width=0.95\textwidth]{metaphoricity-ja}
    \end{figure}
\end{frame}

\begin{frame}
    \frametitle{メタファー性の実験結果II}

     全体的にメタファー性がうまく捉えられているように見受けられるが,
     「情熱を注ぐ」のような慣用的なメタファーについて
     評価値が小さくなってしまう傾向がある.

    \begin{table}\centering
        \footnotesize
        \begin{tabular}{|c|c|c|c|c|c|c|c|c|}
        \hline
            & 愛 & 希望 & 情熱 & 嫉妬 & 不満 & ネコ & 水 \\
        \hline
        \emph{X}に溺れる & \cellcolor{sOrange}\textcolor{white}{2.4} & \cellcolor{sOrange}\textcolor{white}{2.6} & \cellcolor{sBlonde}1.7 & \cellcolor{sDarkRed}\textcolor{white}{3.4} & \cellcolor{sDarkRed}\textcolor{white}{3.0} & \cellcolor{sBlonde}1.9 & 0.3 \\
        \hline
        \emph{X}が沸騰する & \cellcolor{sOrange}\textcolor{white}{2.9} & \cellcolor{sOrange}\textcolor{white}{2.3} & \cellcolor{sOrange}\textcolor{white}{2.5} & \cellcolor{sOrange}\textcolor{white}{2.4} & \cellcolor{sOrange}\textcolor{white}{2.4} & \cellcolor{sOrange}\textcolor{white}{2.1} & 0.0 \\
        \hline
        \emph{X}を注ぐ & \cellcolor{sOrange}\textcolor{white}{2.0} & \cellcolor{sOrange}\textcolor{white}{2.5} & 0.4 & \cellcolor{sOrange}\textcolor{white}{2.9} & \cellcolor{sBlonde}1.9 & \cellcolor{sBlonde}1.7 & 0.6 \\
        \hline
        \emph{X}が満ちる & \cellcolor{sOrange}\textcolor{white}{2.7} & \cellcolor{sOrange}\textcolor{white}{2.1} & \cellcolor{sOrange}\textcolor{white}{2.3} & \cellcolor{sBlonde}1.5 & \cellcolor{sBlonde}1.7 & \cellcolor{sOrange}\textcolor{white}{2.8} & 0.6 \\
        \hline
        \emph{X}が漏れる & \cellcolor{sOrange}\textcolor{white}{2.7} & \cellcolor{sBlonde}1.9 & \cellcolor{sBlonde}1.7 & \cellcolor{sBlonde}1.9 & \cellcolor{sBlonde}1.0 & \cellcolor{sOrange}\textcolor{white}{2.0} & 0.7 \\
        \hline
        \emph{X}が濁る & \cellcolor{sOrange}\textcolor{white}{2.6} & 0.6 & \cellcolor{sOrange}\textcolor{white}{2.2} & \cellcolor{sOrange}\textcolor{white}{2.0} & \cellcolor{sBlonde}1.6 & \cellcolor{sBlonde}1.7 & 0.8 \\
        \hline
            \emph{X}が溢れる & \cellcolor{sBlonde}1.6 & \cellcolor{sBlonde}1.4 & \cellcolor{sBlonde}1.9 & \cellcolor{sDarkRed}\textcolor{white}{3.2} & \cellcolor{sBlonde}1.0 & \cellcolor{sBlonde}1.2 & \cellcolor{sBlonde}1.0 \\
        \hline
        \emph{X}を撒き散らす & \cellcolor{sOrange}\textcolor{white}{2.5} & \cellcolor{sOrange}\textcolor{white}{2.4} & \cellcolor{sOrange}\textcolor{white}{2.0} & \cellcolor{sBlonde}1.8 & \cellcolor{sBlonde}1.1 & \cellcolor{sOrange}\textcolor{white}{2.1} & 0.8 \\
        \hline
        \emph{X}を汲み取る & \cellcolor{sBlonde}1.8 & \cellcolor{sBlonde}1.4 & \cellcolor{sOrange}\textcolor{white}{2.5} & \cellcolor{sOrange}\textcolor{white}{2.4} & \cellcolor{sBlonde}1.5 & \cellcolor{sBlonde}1.6 & 0.1 \\
        \hline
        \end{tabular}
        \caption{メタファー性のヒートマップの一部を抜粋し拡大したもの.}
    \end{table}

\end{frame}

\begin{frame}
    \frametitle{メタファー性の実験結果III}
    \begin{columns}
        \begin{column}{0.55\textwidth}
            \begin{table}[t]
                \centering\footnotesize
                \begin{tabular}{rllc}
                    \toprule%
                    \header{順位} & \header{名詞} & \header{動詞} & \header{得点} \\
                    \midrule%
                    1             & 言葉          & \emph{X}が沸騰する   & 3.9           \\
                    2             & 感情          & \emph{X}に溺れる     & 3.8           \\
                    3             & 絶望          & \emph{X}が溢れる     & 3.7           \\
                    4             & 音            & \emph{X}がしみる     & 3.6           \\
                    %5             & 恐怖          & \emph{X}が流れ出る   & 3.5           \\
                    %5             & 絶望          & \emph{X}がこぼれる   & 3.5           \\
                    5             & 絶望          & \emph{X}を撒き散らす & 3.5           \\
                    %5             & 絶望          & \emph{X}が漏れる     & 3.5           \\[-1pt]
                    \multicolumn{4}{c}{$\vdots$} \\[2pt]
                    %1351          & 水            & \emph{X}が流れ出る   & 0.2           \\
                    %1351          & 水            & \emph{X}が湧く       & 0.2           \\
                    %1355          & 水            & \emph{X}を汲み取る   & 0.1           \\
                    1356          & 声            & \emph{X}をかける     & 0.0           \\
                    1356          & 水            & \emph{X}を啜る       & 0.0           \\
                    1356          & 水            & \emph{X}が沸騰する   & 0.0           \\
                    1356          & 水            & \emph{X}が流れる     & 0.0           \\
                    1356          & 水            & \emph{X}を飲む       & 0.0           \\
                    \bottomrule%
                \end{tabular}
                \caption{メタファー性の上位と下位.}\label{tab:metrank}
            \end{table}
        \end{column}
        \begin{column}{0.45\textwidth}
            \small
            \textbf{傾向}
            \begin{itemize}
                \item 上位にはメタファーらしい表現が並ぶ
                \item 下位には「水」を使った表現が多い
            \end{itemize}

     上位の表現が実際にメタファーかどうかを筆者がMIPVUに従って判定した結果,8/10がメタファー(動詞の語義がメタファー的)と判断された.
     メタファーでなかった2件は「言葉が沸騰する」と「絶望がこぼれる」で,
     いずれも文脈語義が不明であった.

        \end{column}
    \end{columns}
\end{frame}


\begin{frame}
    \frametitle{理解可能性の実験結果I}
    \begin{figure}[h]\centering
        \vfill
        \includegraphics[width=0.95\textwidth]{comprehensibility-ja}
    \end{figure}
\end{frame}

\begin{frame}
    \frametitle{理解可能性の実験結果II}
    \citet{nabeshima2011}の容認度との比較を行ったところ,
    相関係数が0.81で,
    対応する項目の得点差の絶対値の平均が0.64と
    非常に近い結果になった.

    \begin{table}[h]\centering
        \footnotesize
        \begin{tabular}{|c|c|c|c|c|c|c|c|c|}
        \hline
        & 水 & 不満 & 愛 & 情熱 & 怒り & 希望 & 嫉妬 & ネコ \\
        \hline
            \emph{X}が溢れる & \cellcolor{sDarkRed}\textcolor{white}{3.6} & \cellcolor{sDarkRed}\textcolor{white}{3.8} & \cellcolor{sDarkRed}\textcolor{white}{3.7} & \cellcolor{sDarkRed}\textcolor{white}{4.0} & \cellcolor{sDarkRed}\textcolor{white}{3.6} & \cellcolor{sDarkRed}\textcolor{white}{3.3} & \cellcolor{sDarkRed}\textcolor{white}{3.5} & \cellcolor{sOrange}\textcolor{white}{2.1} \\
        \hline
            \emph{X}が満ちる & \cellcolor{sDarkRed}\textcolor{white}{3.8} & \cellcolor{sDarkRed}\textcolor{white}{3.2} & \cellcolor{sDarkRed}\textcolor{white}{3.4} & \cellcolor{sOrange}\textcolor{white}{2.5} & \cellcolor{sDarkRed}\textcolor{white}{3.3} & \cellcolor{sDarkRed}\textcolor{white}{3.6} & \cellcolor{sOrange}\textcolor{white}{2.2} & \cellcolor{sBlonde}1.0 \\
        \hline
            \emph{X}を撒き散らす & \cellcolor{sDarkRed}\textcolor{white}{3.5} & \cellcolor{sDarkRed}\textcolor{white}{3.9} & \cellcolor{sOrange}\textcolor{white}{2.9} & \cellcolor{sOrange}\textcolor{white}{2.2} & \cellcolor{sDarkRed}\textcolor{white}{3.1} & \cellcolor{sOrange}\textcolor{white}{2.2} & \cellcolor{sOrange}\textcolor{white}{2.6} & 0.4 \\
        \hline
            \emph{X}に溺れる & \cellcolor{sDarkRed}\textcolor{white}{3.9} & \cellcolor{sBlonde}1.0 & \cellcolor{sDarkRed}\textcolor{white}{4.0} & \cellcolor{sOrange}\textcolor{white}{2.9} & \cellcolor{sBlonde}1.8 & \cellcolor{sBlonde}1.1 & \cellcolor{sOrange}\textcolor{white}{2.7} & \cellcolor{sOrange}\textcolor{white}{2.5} \\
        \hline
            \emph{X}が漏れる & \cellcolor{sDarkRed}\textcolor{white}{3.9} & \cellcolor{sDarkRed}\textcolor{white}{3.8} & \cellcolor{sOrange}\textcolor{white}{2.1} & \cellcolor{sBlonde}1.6 & \cellcolor{sOrange}\textcolor{white}{2.3} & 0.9 & \cellcolor{sBlonde}1.9 & 0.4 \\
        \hline
            \emph{X}を注ぐ & \cellcolor{sDarkRed}\textcolor{white}{3.7} & \cellcolor{sBlonde}1.3 & \cellcolor{sDarkRed}\textcolor{white}{3.6} & \cellcolor{sDarkRed}\textcolor{white}{4.0} & \cellcolor{sOrange}\textcolor{white}{2.1} & \cellcolor{sOrange}\textcolor{white}{2.3} & \cellcolor{sBlonde}1.4 & 0.1 \\
        \hline
            \emph{X}が濁る & \cellcolor{sDarkRed}\textcolor{white}{3.7} & \cellcolor{sBlonde}1.4 & \cellcolor{sBlonde}1.5 & \cellcolor{sBlonde}1.7 & 0.9 & 0.8 & 0.4 & 0.1 \\
        \hline
            \emph{X}が沸騰する & \cellcolor{sDarkRed}\textcolor{white}{4.0} & \cellcolor{sOrange}\textcolor{white}{2.1} & \cellcolor{sOrange}\textcolor{white}{2.0} & \cellcolor{sOrange}\textcolor{white}{2.1} & \cellcolor{sDarkRed}\textcolor{white}{3.2} & \cellcolor{sBlonde}1.2 & \cellcolor{sOrange}\textcolor{white}{2.0} & 0.3 \\
        \hline
        \end{tabular}
        \caption{理解可能性のヒートマップの一部を抜粋し拡大したもの.}
    \end{table}
\end{frame}

\begin{frame}
    \frametitle{理解可能性の実験結果III}
    \begin{columns}
        \begin{column}{0.55\textwidth}
            \begin{table}[t]
                \centering\footnotesize
                \begin{tabular}{rllc}
                    \toprule%
                    \header{順位} & \header{名詞} & \header{動詞} & \header{得点} \\
                    \midrule%
                    % 1             & 不満          & \emph{X}が溜まる     & 4.0           \\
                    1             & 悲しみ        & \emph{X}が溢れる     & 4.0           \\
                    1             & 情熱          & \emph{X}を注ぐ       & 4.0           \\
                    % 1             & 情熱          & \emph{X}が溢れる     & 4.0           \\
                    1             & 愛            & \emph{X}に溺れる     & 4.0           \\
                    % 1             & 感情          & \emph{X}を汲み取る   & 4.0           \\
                    1             & 水            & \emph{X}ですすぐ     & 4.0           \\
                    1             & 水            & \emph{X}をかける     & 4.0           \\[-1pt]
                    \multicolumn{4}{c}{$\vdots$} \\[2pt]
                    1341          & 睡眠          & \emph{X}が滲む       & 0.1           \\
                    1341          & 絶望          & \emph{X}が澄む       & 0.1           \\
                    1341          & ネコ          & \emph{X}を注ぐ       & 0.1           \\
                    % 1356          & アリ          & \emph{X}がしみる     & 0.0           \\
                    % 1356          & ケーキ        & \emph{X}を注ぐ       & 0.0           \\
                    % 1356          & 岩            & \emph{X}を啜る       & 0.0           \\
                    1356          & 怒り          & \emph{X}を啜る       & 0.0           \\
                    1356          & 憎悪          & \emph{X}ですすぐ     & 0.0           \\
                    \bottomrule%
                \end{tabular}
                \caption{理解可能性の上位と下位.}\label{tab:comprank}
            \end{table}
        \end{column}
        \begin{column}{0.45\textwidth}
            \noindent \textbf{傾向}
            \begin{itemize}
                \item 上位には慣用表現と「水」を使った表現が多い.
                \item 下位には「憎悪ですすぐ」など見慣れない表現が並ぶ.
            \end{itemize}


        \end{column}
    \end{columns}
\end{frame}

\begin{frame}
    \frametitle{新規性の実験結果I}
    \begin{figure}[h]\centering
        \vfill
        \includegraphics[width=0.95\textwidth]{novelty-ja}
    \end{figure}
\end{frame}

\begin{frame}
    \frametitle{新規性の実験結果II}
    理解可能性と対照的な結果になっている.

    \begin{table}\centering
        \footnotesize
        \begin{tabular}{|c|c|c|c|c|c|c|c|c|}
            \hline
 & 水 & 不満 & 愛 & 情熱 & 怒り & 希望 & 嫉妬 & ネコ \\
            \hline
\emph{X}が溢れる & 0.9 & 0.6 & 0.5 & 0.4 & \cellcolor{sBlonde}1.3 & 0.8 & 0.6 & \cellcolor{sOrange}\textcolor{white}{2.4} \\
            \hline
\emph{X}が満ちる & 0.9 & \cellcolor{sBlonde}1.8 & \cellcolor{sBlonde}1.6 & \cellcolor{sBlonde}1.9 & \cellcolor{sBlonde}1.5 & 0.7 & \cellcolor{sOrange}\textcolor{white}{2.6} & \cellcolor{sDarkRed}\textcolor{white}{3.7} \\
            \hline
\emph{X}を撒き散らす & 0.5 & 0.3 & \cellcolor{sOrange}\textcolor{white}{2.7} & \cellcolor{sOrange}\textcolor{white}{2.6} & \cellcolor{sOrange}\textcolor{white}{2.1} & \cellcolor{sOrange}\textcolor{white}{2.6} & \cellcolor{sOrange}\textcolor{white}{2.3} & \cellcolor{sDarkRed}\textcolor{white}{3.9} \\
            \hline
\emph{X}に溺れる & 0.2 & \cellcolor{sDarkRed}\textcolor{white}{3.5} & 0.1 & \cellcolor{sOrange}\textcolor{white}{2.2} & \cellcolor{sOrange}\textcolor{white}{2.7} & \cellcolor{sDarkRed}\textcolor{white}{3.4} & \cellcolor{sBlonde}1.9 & \cellcolor{sBlonde}1.9 \\
            \hline
\emph{X}が漏れる & 0.5 & 0.3 & \cellcolor{sOrange}\textcolor{white}{2.6} & \cellcolor{sOrange}\textcolor{white}{2.7} & \cellcolor{sOrange}\textcolor{white}{2.7} & \cellcolor{sOrange}\textcolor{white}{2.7} & \cellcolor{sDarkRed}\textcolor{white}{3.0} & \cellcolor{sDarkRed}\textcolor{white}{3.7} \\
            \hline
\emph{X}が沸騰する & 0.0 & \cellcolor{sOrange}\textcolor{white}{2.4} & \cellcolor{sDarkRed}\textcolor{white}{3.4} & \cellcolor{sOrange}\textcolor{white}{2.7} & \cellcolor{sBlonde}1.9 & \cellcolor{sDarkRed}\textcolor{white}{3.3} & \cellcolor{sDarkRed}\textcolor{white}{3.3} & \cellcolor{sDarkRed}\textcolor{white}{3.7} \\
            \hline
\emph{X}を注ぐ & 0.5 & \cellcolor{sDarkRed}\textcolor{white}{3.0} & 0.6 & 0.0 & \cellcolor{sOrange}\textcolor{white}{2.6} & \cellcolor{sOrange}\textcolor{white}{2.3} & \cellcolor{sDarkRed}\textcolor{white}{3.3} & \cellcolor{sDarkRed}\textcolor{white}{3.7} \\
            \hline
\emph{X}が濁る & 0.3 & \cellcolor{sDarkRed}\textcolor{white}{3.6} & \cellcolor{sDarkRed}\textcolor{white}{3.0} & \cellcolor{sDarkRed}\textcolor{white}{3.0} & \cellcolor{sDarkRed}\textcolor{white}{3.3} & \cellcolor{sDarkRed}\textcolor{white}{3.0} & \cellcolor{sDarkRed}\textcolor{white}{3.9} & \cellcolor{sDarkRed}\textcolor{white}{3.9} \\
            \hline
        \end{tabular}
        \caption{新規性のヒートマップの一部を抜粋し拡大したもの.}
    \end{table}
\end{frame}

\begin{frame}
    \frametitle{新規性の実験結果III}
    \begin{columns}
        \begin{column}{0.6\textwidth}
            \begin{table}[t]
                \centering\footnotesize
                \begin{tabular}{rllc}
                    \toprule%
                    \header{順位} & \header{名詞} & \header{動詞} & \header{得点} \\
                    \midrule%
                    % 1             & アリ          & \emph{X}がしみる     & 4.0           \\
                    % 1             & ネコ          & \emph{X}を搾り出す   & 4.0           \\
                    1             & ネコ          & \emph{X}が流れ出る   & 4.0           \\
                    1             & 勇気          & \emph{X}が漏れる     & 4.0           \\
                    %1             & 岩            & \emph{X}を啜る       & 4.0           \\
                    1             & 憎悪          & \emph{X}ですすぐ     & 4.0           \\
                    1             & 時間          & \emph{X}が沸騰する   & 4.0           \\
                    1             & 楽しさ        & \emph{X}ですすぐ     & 4.0           \\[-1pt]
                    \multicolumn{4}{c}{$\vdots$} \\[2pt]
                    1353          & 情熱          & \emph{X}を注ぐ       & 0.0           \\
                    %1353          & 時間          & \emph{X}をかける     & 0.0           \\
                    % 1353          & 水            & \emph{X}をかける     & 0.0           \\
                    %1353          & 水            & \emph{X}が沸騰する   & 0.0           \\
                    1353          & 水            & \emph{X}が流れる     & 0.0           \\
                    1353          & 水            & \emph{X}が湧く       & 0.0           \\
                    1353          & 水            & \emph{X}を飲む       & 0.0           \\
                    1353          & 言葉          & \emph{X}をかける     & 0.0           \\
                    \bottomrule%
                \end{tabular}
                \caption{新規性の上位と下位.}\label{tab:novrank}
            \end{table}
        \end{column}
        \begin{column}{0.4\textwidth}
            \textbf{傾向}
            \begin{itemize}
                \item 上位には,「ネコ」など具体的なものを指す名詞を含む表現が多い.
                \item 下位には,水に関する表現と,「情熱を注ぐ」などの慣用句が並ぶ.
            \end{itemize}

        \end{column}
    \end{columns}
\end{frame}

\begin{frame}
    \frametitle{各指標間の関係}
    \begin{table}[t]
        \centering\footnotesize
        \begin{tabular}{lccc}
            \toprule
                         & \header{メタファー性}
                         & \header{理解可能性}
                         & \header{新規性} \\
            \midrule
            \textbf{メタファー性} & 1.0   & −0.19 & 0.28  \\
            \textbf{理解可能性}   & −0.19 & 1.0   & −0.92 \\
            \textbf{新規性}       & 0.28  & −0.92 & 1.0  \\
            \bottomrule
        \end{tabular}
        \caption{各指標間の相関係数.}\label{corr}
    \end{table}
    新規性と理解可能性の間に強い負の相関がある.

    \bigskip

    片方だけで十分なのではないか.

    →どちらも高い表現もあり、そうとも言い切れない.

    (例)「不満を飲む」 (理解可能性:3.3,新規性:2.6)\newline
    \phantom{(例)}「羞恥心が湧く」 (理解可能性:3.1,新規性:2.4)
\end{frame}

\begin{frame}
    \frametitle{総合評価}
    \begin{columns}
        \begin{column}{0.6\textwidth}
            \begin{table}[t]
                \centering\footnotesize
                \begin{tabular}{rllr}
                    \toprule%
                    \header{順位} & \header{名詞} & \header{動詞} & \header{得点} \\
                    \midrule%
                    1             & 空気          & \emph{X}に沈む       & 8.9           \\
                    2             & 気持ち        & \emph{X}が沸騰する   & 8.8           \\
                    3             & 恐怖          & \emph{X}が流れ出る   & 8.7           \\
                    4             & 感情          & \emph{X}が沸騰する   & 8.6           \\
                    4             & 羞恥心        & \emph{X}がこぼれる   & 8.6           \\
                    \multicolumn{4}{c}{$\vdots$} \\[2pt]
                    1356          & ケーキ        & \emph{X}を注ぐ       & 4.0           \\
                    1356          & 水            & \emph{X}が沸騰する   & 4.0           \\
                    1356          & 水            & \emph{X}が流れる     & 4.0           \\
                    1356          & 水            & \emph{X}を飲む       & 4.0           \\
                    1360          & 声            & \emph{X}をかける     & 3.8           \\
                    \bottomrule%
                \end{tabular}
                \caption{総合評価の上位と下位.}\label{tab:ranking}
            \end{table}
        \end{column}
        \begin{column}{0.4\textwidth}
            上位には,あまり見かけないが
            意味を解釈できる表現が並ぶ.

            \bigskip

            例えば,「空気に沈む」は
            「雰囲気によって落ち込む」のように解釈できる.
        \end{column}
    \end{columns}
\end{frame}

\begin{frame}
    \frametitle{総合評価の有効性I}
    総合評価の有効性を確認するためには,
    上位の表現が実際に「よいメタファー」になっているかどうかを検証する必要がある.

    \begin{itemize}
        \item よい→より使いたいと感じる
        \item メタファー→MIPVUの基準で動詞がメタファー的かどうか
    \end{itemize}

\end{frame}

\begin{frame}
    \frametitle{総合評価の有効性II}
    \textbf{よさの評価方法}
    \begin{enumerate}
        \item 総合評価のランキングに基づき,
              すべての表現を上位10\%のグループと下位90\%のグループに分ける.

        \item 各グループからランダムに表現を1つずつ抽出し10個のペアを作る.

        \item 作業者1名\footnote[frame]{%
            言語学の知識をもつ大学院生.}
            に,より使いたいと感じる表現を各ペアから1つ選んでもらう.
            このとき表現がメタファーかどうかは考慮しない.

        \item 最後に上位グループのものが,より使いたい表現に選ばれることを確認する.

    \end{enumerate}
\end{frame}

\begin{frame}
    \frametitle{総合評価の有効性III}
    \begin{table}[t]
        \centering\footnotesize
        \begin{tabular}{llc}
            \toprule%
            \header{上位10\%}
            & \header{下位90\%}
            & \header{上位がより好ましい} \\
            \midrule
            不満を飲む(23)       & 油を汲み取る(1087)    & $\checkmark$ \\
            怒りがこぼれる(6)    & 岩に溺れる(1117)      & $\checkmark$ \\
            羞恥心が溜まる(44)   & 羞恥心を注ぐ(856)     & $\checkmark$ \\
            情報が濁る(106)      & 空気を撒き散らす(212) & $\checkmark$ \\
            悲しみがしみる(32)   & 理解が流れる(721)     & $\checkmark$ \\
            \textbf{楽しさが渦巻く}(81)   & \textbf{不満に漬かる}(1241)    & − \\
            \textbf{言葉が滲む}(14)       & \textbf{恐怖が流れる}(307)     & − \\
            感情を注ぐ(44)       & 意図に漬かる(654)     & $\checkmark$ \\
            不安が流れ出る(44)   & 情熱を汲み取る(165)   & $\checkmark$ \\
            情報に溺れる(23)     & 油が溜まる(1241)      & $\checkmark$ \\
            \bottomrule
        \end{tabular}
        \caption{総合評価とより使いたいと感じる表現の対応.
                括弧内の数字は総合評価の順位である.
                作業者にはどちらが上位か分からないように提示した.}\label{tab:human}
    \end{table}
\end{frame}

\begin{frame}
    \frametitle{総合評価の有効性IV}
    上位グループの境界を上位20\%, 30\%, 40\%, 50\%に変更した場合の検証も同様に行ったが,
    大きな変化は見られなかった.

    \bigskip

    \begin{table}[t]
        \footnotesize
        \begin{tabular}{cc}
            \toprule
            \header{上位と下位の境界} & \header{上位のほうが好ましい} \\
            \midrule
            10\%/90\% & 8/10 \\
            20\%/80\% & 6/10 \\
            30\%/70\% & 6/10 \\
            40\%/60\% & 6/10 \\
            50\%/50\% & 7/10 \\
            \bottomrule
        \end{tabular}
        \caption{境界を変更した場合の上位・下位と好ましさの対応.}
    \end{table}
\end{frame}

\begin{frame}
    \frametitle{総合評価の有効性V}
    \textbf{メタファーであることの評価方法}

    抽出された上位の表現に関して,メタファーであるかどうかを筆者1名が判定した.
    結果,以下の6個の表現を除いた44個の表現がメタファーと判断された.

    \bigskip

    \begin{table}
        \centering\footnotesize
        \begin{tabular}{ll}
            \toprule
            \header{非メタファー表現} & \header{動詞の意味の理解が難しい表現} \\
            \midrule
            においをかける & ?岩が滴る \\
            油に沈む       & ?岩が滲む  \\
            岩を飲む       & ?砂が滴る  \\
            \bottomrule
        \end{tabular}
        \caption{抽出された上位の表現のうちメタファーでなかった表現.}\label{tab:ismet}
    \end{table}

    具体的なものを指す名詞が多い.
    →「名詞の具体性」など総合評価に加えることで改善される可能性がある.
\end{frame}

\begin{frame}
    \frametitle{総合評価のまとめ}
   総合評価は「よいメタファー」を見つけるために有効である.
    \begin{itemize}
        \item よさに関して

            33/50ペアで上位の表現がより使いたいと判断された.
        \item メタファーであることに関して

            44/50ペアで上位の表現がメタファーであると判断された.
    \end{itemize}
\end{frame}

\section{まとめ}
\begin{frame}
    \frametitle{まとめ}
    \addtolength{\leftmargini}{-1\zw}
    \begin{enumerate}
        \item 直喩に限定されないメタファーの生成において,
            生成された表現を評価するための指標を提案した.
        \item クラウドソーシングを用いて各指標について適切に評価できることを示した.
        \item クラウドソーシングの結果から各指標間の関係を示した.
        \item 提案指標が「よいメタファー」の発見に有効であることを示した.
    \end{enumerate}
\end{frame}

\section{今後の課題}
\begin{frame}
    \frametitle{今後の課題}
    \addtolength{\leftmargini}{-1\zw}
    \begin{itemize}
		\item 《感情は水》以外のメタファーで
            同様の評価を行い,指標の一般性を検証する.

        \item 評価指標の改善を検討する.

            (例)「岩が滴る」のような表現を除くために
            「名詞の具体性」を利用するなど.

        \item 生成を行うシステムを開発し,
            その生成結果を今回提案した指標で評価する.
    \end{itemize}
\end{frame}

\section{参考文献}
\begin{frame}[plain,noframenumbering,allowframebreaks]
\frametitle{参考文献}
\begingroup
\footnotesize
    \setbeamertemplate{bibliography item}[triangle]
    \bibliographystyle{jecon}
    \bibliography{nlp}
\endgroup
% https://tex.stackexchange.com/questions/167419/avoid-frame-numbering-in-references-that-span-more-than-one-frame-in-beamer
\end{frame}

\end{document}
