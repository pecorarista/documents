\documentclass[12pt]{ltjsarticle}
\usepackage{lmodern}
\usepackage{amsmath,amssymb}
\usepackage{datetime}
\usepackage{indent}
\usepackage[safe]{tipa}
\usepackage{luatexja-otf}
\usepackage{luatexja-fontspec}
\usepackage[hiragino-pron,deluxe,expert]{luatexja-preset}
\usepackage{polyglossia}
\usepackage{url}
\setmainlanguage{english}
\setmainfont[Ligatures=TeX]{CMU Serif}
\setsansfont[Ligatures=TeX]{DejaVu Sans}
\setmonofont[Ligatures=TeX]{CMU Typewriter Text}
\setotherlanguage{russian}
\ltjsetparameter{jacharrange={-2,-3,-8}}
\setlength{\textwidth}{128mm}
\setlength{\textheight}{195mm}
\setlength{\oddsidemargin}{0mm}
\setlength{\evensidemargin}{0mm}
\setlength{\topmargin}{0mm}
\setlength{\headheight}{0mm}
\setlength{\headsep}{0mm}
\setlength{\footskip}{10mm}
\usepackage{natbib}
\bibliographystyle{jeconcomma}
\usepackage[unicode,hidelinks]{hyperref}
\hypersetup{
  pdfinfo={
    Author={@pecorarista},
    Title={攻殻機動隊 S.A.C. の OP をロシア語で歌いたい},
    CreationDate={D:20150404000000},
    ModDate={\pdfdate},
  }
}
\title{攻殻機動隊S.A.C.のOPをロシア語で歌いたい}
\author{@pecorarista}
\date{2015年4月4日(2015年7月28日更新)}
\pagestyle{plain}
\begin{document}

\maketitle
攻殻機動隊ARISEのTV版が今週から放送されるそうです。楽しみですね。
私が攻殻機動隊という作品に初めて触れたのは2008年ころにレンタルビデオ店で攻殻機動隊STAND ALONE COMPLEXのDVDを借りて見たときでした。
名作と呼ばれるだけあってとても魅力的な話でしたが、それと同じくらいOrigaの歌う不思議な雰囲気のオープニング曲\textit{Inner Universe}\nocite{origa2011}が印象的でした。
そこで自然な流れとして「この曲を歌いたい」という願望が湧き上がってきたのですが、この曲の歌詞の約半分はロシア語で、学習のコストの高さからずっと手をつけられずにいました。
そんな状態でもう約7年(初回放送から約12年!)が経ってしまいましたが、その間にロシア語の教材を買ったり、
言語学の知識を身につけたりしたので 「今ならいけるんじゃないか」と思い、記事を書きながら勉強することにしました。

このような背景で書いているため、間違いが含まれている可能性が高いです。
もし参考にされる方がいらっしゃいましたらご注意ください。間違いを見つけた方は\texttt{pecorarista@gmail.com}までご連絡いただけると助かります。なお、この記事の作成にあたり、著作権を侵害することがないように十分に配慮しましたが、万が一侵害していることがあれば同連絡先にご連絡先ください。

この記事を作成した主な目的は、上記のように自分の勉強のためですが、私と同じく\textit{Inner Universe}の歌詞に興味を持っている方や、インターネットを活用した語学の勉強法に興味がある方の参考にもなるように書いたつもりです。
前提となる専門的な知識は設けたくなかったのですが、どうしても記事が肥大化してしまうので、発音記号 (IPA) は知っている前提で書きました。
もしご存知なければ\href{http://ja.wikipedia.org/wiki/%E5%9B%BD%E9%9A%9B%E9%9F%B3%E5%A3%B0%E8%A8%98%E5%8F%B7}{「国際音声記号 - Wikipedia」}
で記号を調べながら読むといいかと思います。

\section{Ангелы и демоны кружили надо мной}
まずは
\href{http://ja.wikipedia.org/wiki/%E3%83%AD%E3%82%B7%E3%82%A2%E8%AA%9E%E3%82%A2%E3%83%AB%E3%83%95%E3%82%A1%E3%83%99%E3%83%83%E3%83%88}{「ロシア語アルファベット - Wikipedia」}
で、どの文字がどの音を持っているのかを大雑把に見ておきます。
ロシア語のアルファベットはキリル文字で、これはラテン文字や数学・物理学でよく使うギリシャ文字に似ています。
なので雰囲気をつかむだけなら(アラビア文字やタイ文字などと比較すると)難しくないと思います。
辞書は\textit{Russian: A Linguistic Introduction}を買いました。これで語彙・文法については準備が整いました。
今回の目標は歌うことなので、発音が分からなければどうしようもありません。
辞書や文法書にも少し記述がありますが、もう少し詳しいものが欲しいです。
そこで、\href{https://ru.wiktionary.org/wiki}{Wiktionay}は各々の語形の強勢の位置の情報も載っているのでこれもかなりいい情報源になりそうです。
ようやくすべての準備が整ったので歌詞中に出てくる単語を単語帳としてまとめていきます。
以下でその単語帳の内容を紹介します。
\begin{itemize}
\item \begin{russian}ангелы\end{russian} \textipa{/{\textprimstress}ang\super{j}Il\super{j}{\textlowering{\textbari}}/}〈名〉天使;動物・男性・複数・主格 ← \begin{russian}ангел\end{russian}\textipa{/{\textprimstress}ang\super{j}Il/}
    \item \begin{russian}и\end{russian} \textipa{/i/}〈接〉と
    \item \begin{russian}демоны\end{russian} \textipa{/{\textprimstress}d\super{j}em@n{\textlowering{\textbari}}/}〈名〉悪魔;動物・男性・複数・主格 ← \begin{russian}демон\end{russian} \textipa{/{\textprimstress}d\super{j}em@n/}
    \item \begin{russian}кружили\end{russian} \textipa{/krU{\textprimstress}Z{\textbari}l\super{j}I/}〈動〉まわす、まわる(=кружиться);複数・過去 ← \begin{russian}кружить\end{russian}\textipa{/krU{\textprimstress}Z{\textbari}t\super{j}/}
    \item \begin{russian}над\end{russian} \textipa{/n@d/}(第1音節にアクセントのある語の前で \textipa{/n@d, n5d/}、無声子音で始まる語の前で \textipa{/n@t, n5t/}、ある種の子音結合の前で \begin{russian}надо\end{russian} \textipa{/n@d@, n@d5/})〈前〉〜の上方で
    \item \begin{russian}мной\end{russian} \textipa{/mnoj/}〈代〉私;具格 ← \begin{russian}я\end{russian} \textipa{/ja/}
\end{itemize}

\bigskip

発音の表記は、こだわり始めるときりがないので、厳密な音声表記ではなく、音素表記を教科書を適当に参考にしながら少し詳しく書いたものになっています。また、あまり一般的でない記号{\guilsinglleft}\textipa{{\textlowering{\textbari}}}{\guilsinglright}を使っていますが、これは\citep{jones2011}において バー付きのイオタで表されている\textipa{/{\textbari}/}の弱化した音として使っています。

\begin{russian}“ангелы”\end{russian}を例にとって、ロシア語の知識がほとんど無い状態でどのようにして単語帳を作っていったかを説明したいと思います。興味のない方は次の項から読み進めてください。
まずダウンロードした\texttt{dict.opcorpora.txt}をVimなどの適当なテキストエディタで開き、\texttt{АНГЕЛЫ}で検索します。
キーボードの設定は“Russian (Russian Federation)”よりも“Russian (US, phonetic)”\footnote{これらの名称は私の使っている環境 (CentOS 7) での名称です。環境によって名称や設定の方法は異なります。}のほうがキーボードに印字されているラテン文字と押下したときに入力されるキリル文字が同じもののほうがすぐに慣れて使いやすいです。
正規表現が使えるならば\texttt{\^}\texttt{АНГЕЛЫ\textbackslash{}t}のように 後続の表現が行頭に現れることを表す記号\texttt{\^}とタブ文字を表す記号 \texttt{\textbackslash{}t}を使うと早く見つかることがあります。さて、検索すると178653行目に\texttt{АНГЕЛЫNOUN,anim,masc plur,nomn}と書いてあるのが見つかります。\texttt{NOUN}は「名詞」ですね。\texttt{anim}はなんでしょうか。
\href{http://www.ruscorpora.ru/en/corpora-morph.html}{こちら}を参考にしました。こちらによると\texttt{anim}は“animate”であることを表しているようです。この“animate”はアニメグッズを売っているアニメイトではなくて、「生命のある」という意味の形容詞です。ロシア語では名詞を生命のあるものを指す名詞生命のないものを指す名詞に区別します。\citep{uda2009}では動物名詞と非動物名詞という用語を使っています。

\bigskip

\begin{indentation}{2\zw}{0pt}
\noindent ロシア語では名詞を\textbf{動物名詞}と\textbf{非動物名詞}に分類する。人間(および動物)と他のもの(植物をふくむ)を異質なものとしてとらえ、文法上区別するからである。
\end{indentation}

\bigskip

次の項目の\texttt{masc}ですが、これは“masculine”の略で名詞が「男性」であることを表します。ロシア語の名詞の「性」は「男性」「女性」「中性」の合計3つです。名詞の性は、日本語話者にとってはわかりにくい概念ですが、名詞が指すものの自然な性別(父や母など)や語末の文字で決まるもので語形変化に影響を与えるものです。つづく\texttt{plur}はpluralの略で「複数形」を表しています。「単数形」は\texttt{sing} (singular)です。最後の\texttt{nomn}はnominative、つまり「主格」を表しています。ロシア語の格は主格nominative、対格accusative、属格genitive、処格locative、与格dative、具格instrumentalの6つです。6つは多いような気がしますが、日本語の「名詞+格助詞」と同じようなものと考えるとそこまで多い気はしないと思います。\texttt{dict.opcorpora.txt}でグループの最初に載っている語形\begin{russian}“ангелы”\end{russian}で辞書を引き、意味を調べます。「天使」だそうです。確かに、そう言われてみると英語の“angel”に似ています。おそらく同じ語源なのでしょう。発音についてですが、博友社の辞典であれば[\textipa{\'{a}ng’Il}]のように書いてあります。
それ以外の辞典は見出しに\begin{russian}\'{а}нгел\end{russian}のように書いてあります。博友社のほうが詳しくていいのですが、慣れてくれば各々の綴り字が語のどの位置にあるか(語頭か語末かなど)や、強勢の有無によってどんな音になるかわかるので冗長に感じるでしょう。とりあえずこれで単語の1つについて必要な情報がすべて揃いました。同様の作業を 20 回くらい続ければ意味を理解しつつロシア語で歌うことができます。

動詞についても 1 つだけ、詳しく見ておきます。\begin{russian}“кружили”\end{russian}を\texttt{dict.opcorpora.txt}で調べると\texttt{VERB,impf,tran plur,past,indc}と書いてあります。\texttt{VERB}から品詞は動詞だとわかります。\texttt{impf}はimperfective(不完了相)を表します。これと対になる概念として\texttt{perf} (perfective)(完了相)があります。
「相」は言語学の用語で、動作の始まりや終わりなどを表す表現を指します。\texttt{impf/perf}については\citep[p. 151]{cubberley2002}に以下のようなことが書いてあります。

\bigskip

\begin{indentation}{2\zw}{0pt}
The Imperfective is the unmarked partner, used where the aspect is irrelevant (a simple statement of fact) or not being emphasised (the emphasis being elsewhere, most often on the subject)(中略). The Perfective form emphasizes completion and, by implication, a resulting state(中略).
\end{indentation}

\bigskip

このあたりに深入りするとなかなか戻ってこれなそうなので、とりあえず「不完了相が基本らしい」ということだけを理解しました。\texttt{tran/intr}は\texttt{tran} (transitive)が対格補語をとれる動詞(他動詞)、\texttt{intr} (intransitive)がそれ以外の動詞(自動詞)であることを示しています。\texttt{past/pres/futr}はそれぞれ時制が過去past、現在present、未来futureであることを表しています。最後の\texttt{indc}は直説法indicativeであることを示しています。
ロシア語の法にはこの他に、命令法imperativeと仮定法conditional/subjunctiveがあります。
命令法でも仮定法でもないものが直説法です。
自分用の単語帳を作る上で、「不完了」や「直説法」のように基本的なものをいちいち書くのは効率が悪いので、
ここでも省略して行きたいと思います。
ここではそれ以外は省略せずに書きますが、
勉強していくうちに見た目からわかる部分が増えてきて多くを省略できるようになってきました。

\section{Разбивали тернии и звёздные пути}
この文には実は問題があって、歌詞カードには上のように表記してあるのですが、実際は\href{http://d.hatena.ne.jp/cess/20070503}{http://d.hatena.ne.jp/cess/20070503}に書いてあるように \begin{russian}\textit{Рассекали тернии и млечные пути}\end{russian} と歌っているようです。ここでは後者の単語のみを調べます。
\begin{itemize}
\item \begin{russian}рассекали\end{russian} \textipa{/r@s\super{j}s\super{j}I{\textprimstress}kal\super{j}I/}〈動〉切る、(水・波などを)切って進む;複数・過去 ← \begin{russian}рассекать\end{russian}\textipa{/r@s\super{j}s\super{j}I{\textprimstress}kat\super{j}/}
\item \begin{russian}тернии\end{russian}\textipa{/{\textprimstress}t\super{j}ern\super{j}IjI/}
\item \begin{russian}млечные\end{russian} \textipa{/{\textprimstress}ml\super{j}e{\textteshlig}\super{j}n{\textlowering{\textbari}}jI/}
\item \begin{russian}пути\end{russian}\textipa{/pU{\textprimstress}t\super{j}i/} 〈名〉道;非動物・男性・複数・対格 ← \begin{russian}путь\end{russian} \textipa{/{\textprimstress}put\super{j}/}
    \begin{itemize}
        \item \begin{russian}млечный путь\end{russian} 天の川、銀河
    \end{itemize}
\end{itemize}

\bigskip

\section{Не знает счастья только тот,}

\begin{itemize}
    \item \begin{russian}не\end{russian} \textipa{/n\super{j}I/} 〈助詞〉(任意の文成分の直前に立ち、それを否定する)
    \item \begin{russian}знает\end{russian} \textipa{/{\textprimstress}znajIt/} 〈動〉知る;3人称単数・現在 ← \begin{russian}знать\end{russian} \textipa{/{\textprimstress}znat\super{j}/}
    \item \begin{russian}счастья\end{russian} \textipa{/S\super{j}S\super{j}{\ae}s\super{j}t\super{j}j@/} 〈名〉幸福;非動物・中性・単数・属格 ← счастье \textipa{/{\textprimstress}S\super{j}S\super{j}{\ae}s\super{j}t\super{j}jI/}
    \item \begin{russian}только\end{russian} \textipa{/{\textprimstress}tol\super{j}k@/} 〈副〉ただ、〜だけ
    \item \begin{russian}тот\end{russian}\textipa{/tot/} 〈代〉(関係代名詞によって導かれる従属節の先行詞として、従属節の説明する人や事物をさす)(〜ところの)その人・もの
\end{itemize}

\bigskip

\section{Кто его зова понять не смог...}

ロシア語がよくわかっていなかったので、意外なところに属格が出てきてしばらく意味がわかりませんでした。文法書を見ると、否定される他動詞の補語は属格になるということが書いてありました\cite[p. 102]{uda2009}。そう言われると上の\begin{russian}“счастья”\end{russian}も同じ理由ですね。この記述を見つけられなければ大きな誤解をしたまま進めるところでした。なお、この否定される他動詞の補語に属格を用いる決まりですが、20世紀初め以降は、一部で対格が許容されるようになっていて、今では共存している状況らしいです(同書)。
\begin{itemize}
    \item \begin{russian}кто\end{russian} \textipa{/{\textprimstress}kto/} 〈関係代名詞〉〜する人(男性単数扱い);主格
    \item \begin{russian}его\end{russian} \textipa{/jI{\textprimstress}vo/} 〈所有代名詞〉彼の、(男性名詞・中性名詞を指して)その
    \item \begin{russian}зова\end{russian} \textipa{/{\textprimstress}zov@/} 〈名〉呼び声;非動物・男性・単数・属格 ← \begin{russian}зов\end{russian} \textipa{/{\textprimstress}zof/}
    \item \begin{russian}понять\end{russian} \textipa{/p5{\textprimstress}n\super{j}{\ae}t\super{j}/} 〈動〉理解する;完了 ← \begin{russian}понимать\end{russian} \textipa{/p@n\super{j}I{\textprimstress}mat\super{j}/}
    \item \begin{russian}смог\end{russian} \textipa{/{\textprimstress}smok/} 〈動〉できる;完了・男性・単数・過去 ← \begin{russian}мочь\end{russian} \textipa{/{\textprimstress}mo{\textteshlig}\super{j}/}
\end{itemize}

\bigskip

\section{Собой остаться дольше...}
ここで発音がよくわからなかったのが\begin{russian}“остаться”\end{russian}の \begin{russian}-ться\end{russian}の部分です。\citep[p. 150]{jones2011}には

\bigskip

\begin{indentation}{2\zw}{0\zw}
In reflexive infinitives ending in -ться the soft sign has no \textit{phonetic} function.
Such infinitives and the third persons singular and plural of reflexive verbs are pronounced,
at normal conversational speed, with \textbf{ts},
and the letter я does not have the significance of ‘preceding palatalized constant’ which it would normally have after a consonant letter.(中略)
As the speed of pronunciation decreases,
the stop element of \textbf{ts} in such words is prolonged, until, in a slow, precise manner of speaking, such words are pronounced with the sequence \texttt{\textbf{t-s}}.
\end{indentation}

\bigskip

\noindent と書いてあります。
\texttt{\textbf{t-s}}は\textipa{/\texttslig/}のような1つの破擦音ではなく、
\textipa{/t/}と\textipa{/s/}の連続を表します。
\citep{jones2011}ではこの説明を踏まえ\textipa{/{\texttslig}/}と表し、
博友社の辞典は\textipa{/t{\texttslig}/}と表しています。
ここでは前者の表記を使います。

\begin{itemize}
    \item \begin{russian}собой\end{russian} \textipa{/s5{\textprimstress}boj/} 〈再帰代名詞〉自分自身;具格 ← себя \textipa{/s\super{j}I{\textprimstress}b\super{j}a/}
    \item \begin{russian}остаться\end{russian} \textipa{/5{\textprimstress}sta{\texttslig}@/} 〈動〉残る、とどまる;完了 ← \begin{russian}оставаться\end{russian} \textipa{/5st5{\textprimstress}va{\texttslig}@/}\footnote{語頭、または強勢の置かれる直前の \begin{russian}‹a›\end{russian} または \begin{russian}‹o›\end{russian} が\textipa{/5/}と発音されます。}
    \item \begin{russian}дольше\end{russian} \textipa{/{\textprimstress}dol\super{j}S{\textlowering{\textbari}}/} 〈副〉長く;比較級 ← \begin{russian}долго\end{russian}\textipa{/{\textprimstress}dolg@/}
\end{itemize}

\bigskip

ここで具格が出てくるのは、何だか妙な感じがします。\citep[p. 205]{cubberley2002}によると、
コピュラや、英語で言うところの“become”や“remain”のような変化に関する意味を持った動詞(“semi-copulative verb”などと呼ぶ人もいるそうです。)は補語として具格を取るのが普通だということでした。
\begin{russian}“оставаться”\end{russian}はまさにそのsemi-copulative verbでしょう。
これでまた1つ謎が解けました。

\section{...Бесконечный бег...}
どんどんいきます。これは短いですね。
\begin{itemize}
    \item \begin{russian}бесконечный\end{russian} \textipa{/b\super{j}Isk5{\textprimstress}n\super{j}e{\textteshlig}\super{j}n{\textlowering{\textbari}}j/} 〈形〉無限の、いつ終わるかわからない;非生物・男性・単数・主格
    \item \begin{russian}бег\end{russian} \textipa{/{\textprimstress}b\super{j}ek/} 〈名〉疾走、ものごとの早い経過;非生物・男性・単数・主格
\end{itemize}

\bigskip

\section{Пока жива я могу стараться на лету не упасть,}
残すはあと 1 文です。ここまで来たら最後まで頑張りましょう。
\begin{itemize}
    \item \begin{russian}пока\end{russian} \textipa{/p5{\textprimstress}ka/} 〈接〉〜する間
    \item \begin{russian}жива\end{russian} \textipa{/Z{\textlowering{\textbari}}{\textprimstress}va/} 〈形〉生きている;女性・単数・短語尾 ← \begin{russian}живой\end{russian} \textipa{/Z{\textlowering{\textbari}}{\textprimstress}voj/}
    \item \begin{russian}я\end{russian} \textipa{/{\textprimstress}ja/} 〈人称代名詞〉私;主格
    \item \begin{russian}могу\end{russian} \textipa{/m5{\textprimstress}gu/}  〈動〉できる;1人称単数・現在 ← \begin{russian}мочь\end{russian} \textipa{/{\textprimstress}mo{\textteshlig}\super{j}/}
    \item \begin{russian}стараться\end{russian} \textipa{/st5{\textprimstress}ra{\texttslig}@/} 〈動〉努力する
    \item \begin{russian}на\end{russian} \textipa{/n@/} (第 1 音節にアクセントのある語の前で、または \begin{russian}а, о\end{russian} ではじまる語の前では \textipa{/n5/};特定の語結合で慣用的にアクセントをとる場合は \textipa{/{\textprimstress}na/})〈前置詞〉〜の上で
    \item \begin{russian}лету\end{russian} \textipa{/{\textprimstress}l\super{j}I{\textprimstress}tu/} 〈名〉飛行;非動物・男性・単数・処格2 ← \begin{russian}лёт\end{russian} \textipa{/{\textprimstress}l\super{j}ot/}
        \begin{itemize}
            \begin{russian}на лету\end{russian} 飛行中に
        \end{itemize}
    \item \begin{russian}упасть\end{russian} \textipa{/U{\textprimstress}past\super{j}/} 〈動〉落ちる、転落する;完了 ← \begin{russian}падать\end{russian} \textipa{/{\textprimstress}pad@t\super{j}/}
\end{itemize}

\bigskip

\begin{russian}“жива”\end{russian}について調べているとき、「短語尾」という見慣れない用語が目に入ってきました。
文法書を読むと、ロシア語の形容詞には、短語尾と長語尾があり、文字通り語尾の長さに違いがあり、複雑な使い分けが行われているということがわかりました。
大きな違いは、長語尾が限定的にも叙述的にも使える一方で、短語尾は叙述的にしか使えないということです。
形容詞の短語尾形は、英語でいうところの形容詞“alive”のような使い方をするのだ、と私は捉えました。
ここで出てきた“жива”はちょうど、英語の“alive” と同じ意味なので、もしかしたら何かしらの共通の背景があるのかもしれません。

「処格2」は、特定の男性名詞が前置詞\begin{russian}в, на\end{russian}に支配されるときにとる処格の形です。「処格2」という用語は一般的な名称ではありません。そもそも日本語で書かれたロシア語の文法書は通常「処格」ではなく「前置格」という用語を使うことが多いようです。

\section{Не разучиться мечтать... любить...}

前からの続きです。これで本当に最後です。
\begin{itemize}
    \item \begin{russian}разучиться\end{russian} \textipa{/r@zU{\textprimstress}{\textteshlig}\super{j}i{\texttslig}@/} 〈動〉(習ったことを)忘れる、〜できなくなる;完了 ← \begin{russian}разучиваться\end{russian} \textipa{/r5{\textprimstress}zu{\textteshlig}\super{j}Iv@{\texttslig}@/}
    \item \begin{russian}мечтать\end{russian} \textipa{/m\super{j}I{\textteshlig}\super{j}{\textprimstress}tat\super{j}/} 〈動〉空想する
    \item \begin{russian}любить\end{russian} \textipa{/l\super{j}U{\textprimstress}b\super{j}it\super{j}/} 〈動〉愛する
\end{itemize}

\bigskip

\section*{まとめ}

ロシア語は、文字以外ほとんどわからない状態からのスタートでしたが、どうにか1週間程度で歌詞の発音と大まかな内容をつかむことができました。
多少の音声の知識や言語学の知識は必要になったものの、これらはWikipediaなどで、音声ファイル付きで勉強することができます。
もし、今後、何らかの言語を急いで習得しなければならなくなったとき、これらの知識があると非常に重宝します。
ただ、これらさえあれば、誰にも教わらずに言語が習得できるかというと、そんなことはありません。
誰かに教わらずに(効率的に)語学の勉強をするにはOpenCorporaやWiktionaryのようなWeb上の資源が欠かせません。
言語を習得するためのコストは(話者がある程度多い言語に限定されてしまいますが)ここ数年で明らかに小さくなっています。
もしもあなたが、アニメ、音楽等の趣味の中で、意味や発音を知らないままにしている表現があれば、今こそそれを調べるための絶好の機会だと思います。\nocite{rusdict}

\bigskip
\bibliography{ref}
\end{document}
